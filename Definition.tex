%!TEX encoding = UTF-8 Unicode
\chapter{Einleitung}
\chapter{Consulting Begriffe}
\section{Begriff}
Als Synonyme zur Unternehmensberatung werden häufig auch die Begriffe “Consulting”
oder “management consulting” verwendet. Im deutschen Duden wird Consulting als Beratung; Beratertätigkeit (besonders in der Wirtschaft) angegeben. Der Begriff Consultung ist daher vollständig eingedeutscht und es wird sich explizit auf wirtschaftliche Beratung bezogen. 
In dieser Arbeit wird sich allerdings hauptsächlich auf den Begriff Unternehmensberatung bezogen, welcher im wissenschaftlichen Umfeld eher Bestand hat .
In der Literatur als auch in der Umgangsprache scheint es zwar einen generellen Konsens zu geben was den Begriff Unternehmensberatung angeht. 
Jedoch werden diese teils aus verschiedenen Perspektiven betrachtet. Häufig wird eine funktionale Perspektive als Ausgangspunkt verwendet, welche den Prozess der Unternehmensberatung als Tätigkeit beschreibt. Es gibt jedoch auch vereinzelt institutionelle Herangehensweisen zur Definition, welche die Unternehmensberatung ausführende Organisation selbst fokussiert. Da beide Begriffe in der Literatur als auch in der Umgangsprache öfter auftauchen soll hier in die beiden Arten funktionaler -und institioneller Begriff unterschieden werden In der nachfolgenden Arbeit der Begriff Unternehmnesberatung im funktionalen Sinne verwendet oder auch als Consulting bezeichnet.

\section{Funktionaler Begriff}

Eine Bestätigung der Uneinheitlichkeit eines Begriffes liefert Niessen \cite[10]{nissen2007consulting}. Er verweist bereits auf die uneinheitliche Begriffsdefinition in der wissenschaftlichen Literatur. \cite[7]{ernst2002evaluation}. Die Ursache liegt laut Ernst \cite[10]{ernst2002evaluation} in der fragmentierten Forschungsgemeinschaft, welche unterschiedliche Untersuchungsziele und Abgrenzungszwecke heranzieht. 
Jedoch gibt es einen vermeintlichen Konsens indem sich Definitionen in etwas überschneiden. Einig ist sich Literatur in  zumeist darüber, dass die Unternehmensberatung eine eigenverantwortlich, zeitlich befristet, auftragsindividuell und zumeist gegen Entgelt erbrachte professionelle Dienstleistung \cite[14]{Lippold201309}, die durch eine oder mehrere, im allgemeinen fachlich dazu befähigte und von den beratenen Klienten hierarchisch unabhängige Personen durchgeführt wird, mit dem Ziel zumeist betriebswirtschaftliche Problemstellungen eines Klienten zu identifizieren und zu analysieren. Dabei kann eine Handlungempfehlung erarbeitet und dem Klienten bei der Planung, Erarbeitung der Lösung und Umsetzung geholfen werden.  \cite[15]{nissen2007consulting}

\section{Institutioneller Begriff}

Einen institutionellen Begriff liefert Bamberger \cite[16]{bamberg2008strategische}
Er stellt Unternehmensberatungen selbst als Organisation dar. Sie haben Ziele, Strategien, Managementsysteme, Wertschöpfungsketten, Geschäftsprozesse, Organisationsstrukturen und eine Organisationskultur. Sie können unterschiedliche Geschäftsmodelle aufweisen. Diese Art der Betrachtung ist durchaus sinnvoll, da sich die Art der Unternehmung folglich auch auf den Prozess selbst auswirkt. Insbesondere im Vergleich verschiedener Beratungsunternehmen ist die Verwendung dieses Begriff folglich ebenso sinnvoll.

\section{Inhouse-Beratung}
Der Begriff Inhouse Consulting wird häufig auch als interne Beratung bezeichnet. \cite[150]{ReinekeBock200709}
Im Grunde genommen verfolgen die internen Beratungen die gleichen Ziele wie die Externen. Die Mehrheit der Inhouse-Beratungen besteht entweder als GmbH oder selbständige Stabsstelle, die in den allermeisten Fällen direkt beim Vorstand bzw. der Geschäftsführung angegliedert sind. \cite[14]{B2_InhouseConsulting}
Es gibt dennoch einige zusätzliche Aspekte die sich auf die interne Beratung nicht unwesentlich auswirken. Diese Aspekte liegen zum einen in der Entstehung von Inhouse Beratungen. Diese haben meistens einen längeren evolutorischen Transformationsprozess hinter sich, der aus verschiedenste Sondersituationen wie Restrukturierungen, Zentralisierungen,Fusionen etc. einhergeht. \cite[160]{Lippold201309}
Für Unternehmen mit einem Beratungsbedarf stellt sich die Entscheidungsfrage, ob eine interne Beratung etabliert oder eine externe Beratung angeheuert werden soll.
Aufgrund der unmittelbaren Nähe zum Top-Management und zu den Abteilungen gibt es natürlich einige Vorteile für eine interne Beratung hinsichtlich der Kommunikationswege.
Diese Tatsache verspricht natürlich Kosteneinsparungen und Synergieeffekte. Eine Erörterung der Vor -und Nachteile soll hier allerdings nicht erfolgen, da der primäre Fokus auf Erfassung der Daten liegt und nicht auf dessen Bewertung.

\section{Extern}
 to-do
\chapter{Consulting-Systematiken}
allgemeines Geblubber warum es so schwer ist Systematiken zu finden ;)
\section{BDU-Systematik}
In einigen Literaturquellen \cite[54]{Lippold201309},\cite[4]{nissen2007consulting} wird eine weitere Einteilung der Beratungsfelder angegeben. Diese wird auch als BDU-Systematik bezeichnet (BDU: Bund Deutscher Unternehmensberater). Der BDU verwendet die Einteilung als Grundlage für seine statistischen Erhebungen.Deswegen ist sie trotz ihrer Schwächen (nächster Abschnitt) relevant.
Die BDU Einteilung unterteilt in: Strategieberatung, Organisations- und Prozessberatung, IT-Beratung und Human-Ressource Beratung. Eine klare Definition der einzelnen Bestandteile der 4 Bereiche wird vom BDU nicht angegeben. Einige Kritikpunkte an dieser Einteilung werden von (\cite[54]{Lippold201309} aufgeführt. So wird kritisiert: : \glqq dass sich Organisations- und Prozessberatung nicht oder nur sehr schwer von der IT-Beratung trennen l\"asst \grqq. (\cite[54]{Lippold201309} . Dies lässt sich leicht verstehen indem man sich die Abhängigkeiten von IT und Organisation verdeutlicht. Eine klare Einteilung in diese zwei Bereiche erscheint schwierig. Eine weitere „wesentliche Schwäche“ (\cite[54]{Lippold201309}  wird deutlich wenn man das 4. Beratungsfeld die Human Ressource betrachtet. Laut (\cite[54]{Lippold201309} wird sie als einzige der funktionalen Beratungsarten erwähnt. Die anderen funktionalen Beratungsarten (z.B.Marketingberatung, Controlling-Beratung, Logistik- Beratung etc.) finden überhaupt keine Erwähnung.
Trotz dieser wesentlichen Schwächen sollen hier eine kurze Erklärung der einzelnen Beratungsfelder folgen. Weiterhin wird der Versuch einer Abgrenzung der Begriffe untereinander unternommen. Da die Human-Ressource Beratung zu den funktionalen Beratungsfelder gehört, werden im folgenden nur die 3 verbleibenden Kernberatungsfelder erklärt. Auf eine Erklärung der funktionalen Beratungsfelder wird verzichtet. 

\subsection{Strategieberatung}
Laut (B4, S.60) betrifft die Strategieberatung den „Kernbereich aller Unternehmensaktivitaeten, die Unternehmensstrategie“ . Zu den wichtigsten Ansprechpartnern zählen daher „Vorstände und Geschäftsführer“ . Außerdem ist die Unternehmensstrategie „mit großer Unsicherheit behaftet“ . Daher gilt es „[...] zu antizipieren“.  Gegenstand der Strategieberatung sind „Zielkunden,Leistungsversprechen und Geschäftsmodelle“. Dabei werden „verschiedene Auffassungen über die Weiterentwicklung des Unternehmens“ diskutiert.
Die Anlässe für die Strategieberatung sind vielfältig. Es kann um das Neuentwickeln,Ändern,Weiterentwicklung, Verifizierung und Umsetzung von Strategien handeln.
Die Aufgaben der Strategieberatung sind vielfältig dazu zählen z.B. Bestandsaufnahme,Problemerkennung und Identifizierung ,Auswahl relevanter Informationen, Hypothesenentwicklung, Analyse und Bewertung, Szenarioentwicklung, Entscheidungsvorbereitung, Umsetzungsplanung.

\subsection{Organisations- und Prozessberatung}
Laut \cite[63]{Lippold201309} beschäftigt sich die Organisations- und Prozessberatung mehr \glqq mit Fragen der Aufbau- oder Ablauforganisation sowie Prozessen\grqq und setzt dabei auf eine \glqq bestehende oder neu erarbeitete Strategie eines Unternehmens auf \grqq. Das Ziel \glqq dabei ist die Leistungs- und Anpassungsfähigkeit der Kundenunternehmen durch die Gestaltung oder Neugestaltung der Strukturen und Prozesse zu verbessern \grqq. Es geht darum Strukturen und Prozesse \glqq effektiver und/oder effizienter \grqq zu gestalten. Im Gegensatz zur Strategieberatung bewegt sich die Organisations- und Prozessberatung daher eher auf der Umsetzungsebene obwohl eine Abgrenzung generell schwer fällt. \cite[63]{Lippold201309} unterscheidet außerdem verschiedene Arten der Organisations- und Prozessberatung.
Es wird unterschieden in die gutachterliche Beratung die \glqq vornehmlich dem Wissenstransfer und der Erkenntnisvermittlung \grqq dient. So können \glqq wissenschaftliche Erkenntnisse in das Kundenunternehmen transferiert werden.\grqq . Die Expertenberatung dient dazu einen \glqq Problemlösungsprozess \grqq anzustoßen. Es wird im Gegensatz zur gutachterlichen Beratung wird hier auch die Umsetzung beachtet. Die dritte Art ist dies Organisationsentwicklung. Dort ist der Berater \glqq eher passiv\grqq. Die Mitarbeiter des Unternehmens sollen nach einer Anlernphase ihr Unternehmen selbst \glqq entwickeln \grqq . Die letzte Art, die systematische Beratung ist aus der neueren Systemtheorie entstanden. Der Kunde wird unterstützt \glqq bei seiner Selbstreflektion\grqq.

\subsection{IT-Beratung}
	\subsubsection{Grund für den Zuwachs von IT-Beratung}
	
		Ein wesentlicher Grund für den starken Wachstum von den IT-Beratungsleistungen liegt in der Verbreitung der Informationstechnologie in Unternehmen. Zahlreiche Statistiken und Studien bestätigen, dass der Einsatz von modernen IT-Technologien die Arbeitsproduktivität erhöht, Geschäftsmodelle im Unternehmen automatisiert und verbessert. Rund 60% der Beschäftigten in Deutschland  erledigen ihre Arbeit am Computer.(Quelle: Z10)
		Fachbereich IT führt heute nicht nur zur Optimierung der Geschäftsprozesse sondern ist als Business Partner bei erfolgreichen Unternehmen zu verstehen. Denn automatisierte Geschäftsprozesse auf Basis moderner Technologien wie Social Media, RFID, Big Data oder Cloud  ermöglichen es dem Betrieb, ihre Positioniereung weiter auszubauen. (Quelle Z12 )
		Aber nur blinde Investition in neue Technik führt nicht zum Geschäftserfolg. Viel wichtiger ist es zu Wissen an welcher Stelle im Unternehmen welche Technik eingesetzt werden soll um die Geschäftsprozesskette zu optimieren. IT-Technik ist dabei ein bedeutender Wertschöpfungsfaktor zum Zweck der Prozessverbesserung.
		Dadurch wird ersichtlich wie groß die Rolle von IT-Beratungleistungen heutzutage  für modernen Unternehmen ist. Ein wesentlicher Baustein dieser Arbeit liegt daran, den Begriff der IT-Beratung  zu definieren sowie die Abgrenzung zur klassischen Unternehmensberatung und anderen Beratungsarten  zu klären.
		Genauso wie der Begriff “Consulting” ist der Begriff “IT-Consulting” auch vollständig eingedeutscht und bezieht sich direkt auf IT-Beratung. Beide Begriffe: “IT-Consulting” und “IT-Beratung” werden in dieser Arbeit als Synonyme verwendet.
		\subsubsection{Definition aus dem Gabler Lexikon Unternehmensberatung}
		IT-Consulting: IT-Beratung, Consulting von Unternehmen bei der Gestaltung von Prozessen, die durch Informationstechnologie (IT) unterstützt werden, sowie bei der Einführung von neuen IT-Systemen und -Anwendungen. Darüber hinaus unterstützen viele IT-Con­sultants die Unternehmen auch in den Bereichen Systementwicklung und -integration.(Quelle: B9)
		\subsubsection{ Definition aus dem Online-Statistik-Portal statista.com}
			Unter IT-Consulting wird die professionelle Beratung von Unternehmen und Projekten bei der Entwicklung, Installation und Weiterführung von IT-Systemen verstanden. In Deutschland wird IT-Consulting auch als IT-Beratung bezeichnet. Der spezielle Bereich IT-Consulting ist ein wachsender Wirtschaftszweig innerhalb der gesamten IT-Branche. Im Allgemeinen zählt man den Bereich IT-Consulting zur Wirtschaftsbranche der Unternehmensberatung.(Quelle: Z2)
		\subsubsection{Zusammenfassung}
			
			
			Auf den ersten Blick kann man leicht den Begriff IT-Beratung bestimmen. IT-Beratung ist eine Beratung von Unternehmen in Fragen der Informationstechnologie. Probleme beginnen bereits bei der Definition von IT. Informationstechnologie ist weit gefasstes Thema mit vielen Dienstleistungen (Integration von individueller Software, Customizing von Standardsoftware usw.), die für die IT-Beratung sinnvoll sind. Ziel ist dabei die Optimierung von Geschäftsprozessen und IT-Infrastruktur mit Hilfe von IT.
			Ein 2.Aspekt, der für für den einheitlichen Begriff der IT-Beratungist ein Hinterniss ist, verbirgt sich hinter der Klassifikation einer Beratung. Es lassen sich sehr viele Beratungsthemen unterscheiden: Geschäftsstrategien und -Optimierungen, Personalausbildung, Prozessgestaltung und Prozessverbesserung, Systemimplementierung, Organisationsberatung, Sektor- und branchenorientierte Beratung, Markt- und Rechtsberatung sowie Spezialthemen wie
			Regulierung/Deregulierung,  Privatisierung oder Kulturanalyse und -anpassung, Prozessberatung und  interkulturelle Beratung (Quelle B9).
			Darüber hinaus unterstützen viele IT-Consultants die Unternehmen auch in den Bereichen Systementwicklung und -integration. Dabei orientieren sich die Berater stark an Modellen und Methoden der (Wirtschafts-) Informatik [Quelle Z11].
			

\section{Lünendonk Systematik}
Eine weiterer Unterteilungsversuch für die Consulting Branche wurde von der Marktforschungsfirma Lünendonk GmbH verfasst.
Diese Systematik wird auch in der wissenschaftlichen Literatur \cite[56]{Lippold201309} verwendet. Lünendonk veröffentlicht
auch die sogenannten Lünendonk Listen. Diese zeigen im Bezug auf Deutschland eine Art Topliste der besten Firmen (bezogen auf bestimmte Kriterien).
Die Listen sind auf der Website der Firma im Shop kostenlos herunterladbar \cite {topBITP} . Weiterhin verkauft Lünendonk in diesem Shop verschiedenste Studien u.a.
auch zu IT Consulting Themen.

Die für diese Arbeit relevante Systematik stammt aus: \cite[56]{Lippold201309}. Sie unterteilt den IT Beratungsprozess in 6 Teile :  Strategieberatung ,Organisations- und Prozessberatung, IT-Beratung (Prozesse, Technologien, Infrastruktur), IT-Systemintegration, IT-System-Betrieb, Betrieb kompletter Geschäftsprozesse (BPO). Je nach der Abdeckung dieser 6 Prozesse werden dann die verschiendenen Firmen der Branche zugeordnet. So beschäftigt sich die Strategieberatung nach Lünendonk fast nur mit Strategieberatung und Organisations- und Prozessberatung. Es existiert auch noch eine kleine Überlappung mit der IT-Beratung d.h. ein Teil der Strategieberatungen beschäftigt sich auch damit. Die Kategorie [2] IT- Beratungs und Systemintegrationsunternehmen ist eher technisch orientiert, berät aber aufgrund der Abhängigkeiten von Organisations- und Prozessberatung und der anschließenden IT Beratung auch im ersten Bereich mit. Die oben erwähnten Lünendonk Listen greifen auf diese Systematik zurück und präsentieren die jeweiligen \glqq Toplisten \grqq für diese Unternehmensarten. Die Lünendonk Systematik könnte man als zu IT-Lastig kritisieren \cite[56]{Lippold201309} . Für die vorliegende Arbeit ist sie aber aufgrund des gewählten Schwerpunktes
IT- Consulting gut geeignet.

\subsection{Full-Service-Beratungen}
Definition Full-Service-Beratung
Der Begriff Full-Service wird in der Literatur nicht einheitlich bezeichnet. Es gibt für den betreffenden Sachverhalt mehrere Synonyme, welche zum Teil weiter gefasst sind und andere Aspekte enthalten. Das häufig aufzufindenste Synonym in diesem Bereich ist der Begriff Full-Service Provider.
Der sog. \glqq Full-Service-Provider \grqq zielt darauf ab die kompletten Anforderungen eines Kunden-Teilsegmentes abzudecken. \cite[124]{WeillVitale200106} Dies beinhaltet folglich das gesamte Spektrum von der Strategie bis zur Umsetzung. Da Full-Service-Provider eine Komplettlösung anbieten, schließt das natürlich die Leistungen eines Content, Application und Service Providers mit ein. \cite[83]{Thalmann200708}
Es wird in diesem Zusammenhang auch die Bezeichnung Business Innvoation / Transformation Partner (BITP) verwendet, welches wiederum mit dem Begriff BPO (Business Process Outsourcing) verwandt ist. Hier liegt der Schwerpunkt vorallem auf eine langfrististe und strategische Übernahme von ganzen Unternehmens -oder Produktsegmenten, welche vor allem Outsourcing-Charakter hat.\cite[163]{Pohland200908} Häufig ist dort aufgrund der Umsetzungsnähe ein großer IT Anteil aufzufinden. Aufgrund dessen fallen einige der Unternehmen die sich als Full-Service-Provider auch in die Gruppe der BITP. Dies lässt sich an den einschlägigen Lünedonk-Rankings beobachten. Dort gibt es Unternehmen die sowohl im Lünedonk Top 15 Ranking für BITP auftauchen, als auch im Top 25 Ranking der IT-Beratungen.\cite {topBITP} \cite {topITB}

\subsection{Business Prozess Outsourcing }
Generell gibt es keine einheitlich zur Begriffsdefinition von Business Prozess Outsourcing (BPO). 
Einen mehr betriebswirtschaftlichen Begriff liefert Tüfekciler:
BPO beschreibt die den Transfer des Management und der Durchführung ein oder mehrerer kompletter Prozesse oder Geschäftsbereiche.  \cite[14]{tuefekciler2011human} 
Häufig hört man den Begriff BPO in Zusammenhang mit IT-Prozessen. 

Eine Zusammenfassung als Liste mit wesentlichen Merkmalen aus mehreren IT-orientierten BPO Begriffsansätzen, liefert Mauchle \cite[6]{mauchle2012business}:

- Vertragliche Vereinbarung zwischen den beteiligten Parteien
- Auslagerung von spezifischen Prozessen an ein anderes Unternehmen oder einen anderen Unternehmensbereich
- Übergang der operativen Kontrolle und Prozesssteuerung
- hohe IT-Intensität

Mauche versteht unter dem Begriff Business Prozess Outsourcing eine vertraglich geregelte Auslagerung eines typischerweise IT-intensiven Geschäftsprozesses an einen externen Dienstleister, welcher den Prozess fortan unter eigener operativer Steuerung und Kontrolle ausführt.
\cite[3]{mauchle2012business}

Typische BPO Prozesse sind laut Halvey Unternehmensbereiche wie Finanzen und Buchhaltung, Investitionsverwaltung, Personalverwaltung und Logistik. Aber auch kleiner oder noch größere Unternehmensbereiche werden in der Literatur als BPO Prozesse bezeichnet. \cite[4]{halvey2007business}
 







