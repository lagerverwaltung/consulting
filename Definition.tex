\chapter{Definition}
\section{Einleitung}
Einleitender Text

\section{Begriff}
Als Synonyme zur Unternehmensberatung werden häufig auch die Begriffe “Consulting”
oder “management consulting” verwendet. Im deutschen Duden wird Consulting als Beratung; Beratertätigkeit (besonders in der Wirtschaft) angegeben. Der Begriff Consultung ist daher vollständig eingedeutscht und es wird sich explizit auf wirtschaftliche Beratung bezogen. 
In dieser Arbeit wird sich allerdings hauptsächlich auf den Begriff Unternehmensberatung bezogen, welcher im wissenschaftlichen Umfeld eher Bestand hat .
In der Literatur als auch in der Umgangsprache scheint es zwar einen generellen Konsens zu geben was den Begriff Unternehmensberatung angeht. 
Jedoch werden diese teils aus verschiedenen Perspektiven betrachtet. Häufig wird eine funktionale Perspektive als Ausgangspunkt verwendet, welche den Prozess der Unternehmensberatung als Tätigkeit beschreibt. Es gibt jedoch auch vereinzelt institutionelle Herangehensweisen zur Definition, welche die Unternehmensberatung ausführende Organisation selbst fokussiert. Da beide Begriffe in der Literatur als auch in der Umgangsprache öfter auftauchen soll hier in die beiden Arten funktionaler -und institioneller Begriff unterschieden werden In der nachfolgenden Arbeit der Begriff Unternehmnesberatung im funktionalen Sinne verwendet oder auch als Consulting bezeichnet.

\subsection{Funktionaler Begriff}

Eine Bestätigung der Uneinheitlichkeit eines Begriffes liefert Niessen \cite[10]{nissen2007consulting}. Er verweist bereits auf die uneinheitliche Begriffsdefinition in der wissenschaftlichen Literatur. \cite[7]{ernst2002evaluation}. Die Ursache liegt laut Ernst \cite[10]{ernst2002evaluation} in der fragmentierten Forschungsgemeinschaft, welche unterschiedliche Untersuchungsziele und Abgrenzungszwecke heranzieht. 
Jedoch gibt es einen vermeintlichen Konsens indem sich Definitionen in etwas überschneiden. Einig ist sich Literatur in  zumeist darüber, dass die Unternehmensberatung eine eigenverantwortlich, zeitlich befristet, auftragsindividuell und zumeist gegen Entgelt erbrachte professionelle Dienstleistung \cite[14]{Lippold201309}, die durch eine oder mehrere, im allgemeinen fachlich dazu befähigte und von den beratenen Klienten hierarchisch unabhängige Personen durchgeführt wird, mit dem Ziel zumeist betriebswirtschaftliche Problemstellungen eines Klienten zu identifizieren und zu analysieren. Dabei kann eine Handlungempfehlung erarbeitet und dem Klienten bei der Planung, Erarbeitung der Lösung und Umsetzung geholfen werden.  \cite[15]{nissen2007consulting}

\subsection{Institutioneller Begriff}

Einen institutionellen Begriff liefert Bamberger \cite[16]{bamberg2008strategische}
Er stellt Unternehmensberatungen selbst als Organisation dar. Sie haben Ziele, Strategien, Managementsysteme, Wertschöpfungsketten, Geschäftsprozesse, Organisationsstrukturen und eine Organisationskultur. Sie können unterschiedliche Geschäftsmodelle aufweisen. Diese Art der Betrachtung ist durchaus sinnvoll, da sich die Art der Unternehmung folglich auch auf den Prozess selbst auswirkt. Insbesondere im Vergleich verschiedener Beratungsunternehmen ist die Verwendung dieses Begriff folglich ebenso sinnvoll.
 
\section{Beratungsarten}
\subsection{Inhouse-Beratung}
Der Begriff Inhouse Consulting wird häufig auch als interne Beratung bezeichnet. \cite[150]{ReinekeBock200709}
Im Grunde genommen verfolgen die internen Beratungen die gleichen Ziele wie die Externen. Die Mehrheit der Inhouse-Beratungen besteht entweder als GmbH oder selbständige Stabsstelle, die in den allermeisten Fällen direkt beim Vorstand bzw. der Geschäftsführung angegliedert sind. \cite[14]{B2_InhouseConsulting}
Es gibt dennoch einige zusätzliche Aspekte die sich auf die interne Beratung nicht unwesentlich auswirken. Diese Aspekte liegen zum einen in der Entstehung von Inhouse Beratungen. Diese haben meistens einen längeren evolutorischen Transformationsprozess hinter sich, der aus verschiedenste Sondersituationen wie Restrukturierungen, Zentralisierungen,Fusionen etc. einhergeht. \cite[160]{Lippold201309}
Für Unternehmen mit einem Beratungsbedarf stellt sich die Entscheidungsfrage, ob eine interne Beratung etabliert oder eine externe Beratung angeheuert werden soll.
Aufgrund der unmittelbaren Nähe zum Top-Management und zu den Abteilungen gibt es natürlich einige Vorteile für eine interne Beratung hinsichtlich der Kommunikationswege.
Diese Tatsache verspricht natürlich Kosteneinsparungen und Synergieeffekte. Eine Erörterung der Vor -und Nachteile soll hier allerdings nicht erfolgen, da der primäre Fokus auf Erfassung der Daten liegt und nicht auf dessen Bewertung.

\subsection{Full-Service-Beratungen}
Definition Full-Service-Beratung
Der Begriff Full-Service wird in der Literatur nicht einheitlich bezeichnet. Es gibt für den betreffenden Sachverhalt mehrere Synonyme, welche zum Teil weiter gefasst sind und andere Aspekte enthalten. Das häufig aufzufindenste Synonym in diesem Bereich ist der Begriff Full-Service Provider.
Der sog. “Full-Service-Provider” zielt darauf ab die kompletten Anforderungen eines Kunden-Teilsegmentes abzudecken. \cite[124]{WeillVitale200106} Dies beinhaltet folglich das gesamte Spektrum von der Strategie bis zur Umsetzung. Da Full-Service-Provider eine Komplettlösung anbieten, schließt das natürlich die Leistungen eines Content, Application und Service Providers mit ein. \cite[83]{Thalmann200708}
Es wird in diesem Zusammenhang auch die Bezeichnung Business Innvoation / Transformation Partner (BITP) verwendet, welches wiederum mit dem Begriff BPO (Business Process Outsourcing) verwandt ist. Hier liegt der Schwerpunkt vorallem auf eine langfrististe und strategische Übernahme von ganzen Unternehmens -oder Produktsegmenten, welche vor allem Outsourcing-Charakter hat.\cite[163]{Pohland200908} Häufig ist dort aufgrund der Umsetzungsnähe ein großer IT Anteil aufzufinden. Aufgrund dessen fallen einige der Unternehmen die sich als Full-Service-Provider auch in die Gruppe der BITP. Dies lässt sich an den einschlägigen Lünedonk-Rankings beobachten. Dort gibt es Unternehmen die sowohl im Lünedonk Top 15 Ranking für BITP auftauchen, als auch im Top 25 Ranking der IT-Beratungen.\cite {topBITP} \cite {topITB}

\subsection{Business Prozess Outsourcing }
Generell gibt es keine einheitlich zur Begriffsdefinition von Business Prozess Outsourcing (BPO). 
Einen mehr betriebswirtschaftlichen Begriff liefert Tüfekciler:
BPO beschreibt die den Transfer des Management und der Durchführung ein oder mehrerer kompletter Prozesse oder Geschäftsbereiche.  \cite[14]{tuefekciler2011human} 
Häufig hört man den Begriff BPO in Zusammenhang mit IT-Prozessen. 

Eine Zusammenfassung als Liste mit wesentlichen Merkmalen aus mehreren IT-orientierten BPO Begriffsansätzen, liefert Mauchle \cite[6]{mauchle2012business}:

- Vertragliche Vereinbarung zwischen den beteiligten Parteien
- Auslagerung von spezifischen Prozessen an ein anderes Unternehmen oder einen anderen Unternehmensbereich
- Übergang der operativen Kontrolle und Prozesssteuerung
- hohe IT-Intensität

Mauche versteht unter dem Begriff Business Prozess Outsourcing eine vertraglich geregelte Auslagerung eines typischerweise IT-intensiven Geschäftsprozesses an einen externen Dienstleister, welcher den Prozess fortan unter eigener operativer Steuerung und Kontrolle ausführt.
\cite[3]{mauchle2012business}

Typische BPO Prozesse sind laut Halvey Unternehmensbereiche wie Finanzen und Buchhaltung, Investitionsverwaltung, Personalverwaltung und Logistik. Aber auch kleiner oder noch größere Unternehmensbereiche werden in der Literatur als BPO Prozesse bezeichnet. \cite[4]{halvey2007business}
 







