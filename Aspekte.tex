%!TEX encoding = UTF-8 Unicode
\chapter{Aspekte}
\chapter{Markt}
\section{Einleitung}
Im folgenden Kapitel soll ein grober Überblick und eine Einleitung zum IT-Consulting Markt gegeben werden. 
Ziel einer solchen Recherche sowohl Informationen zu sammeln, die strategische Möglichkeiten für Expansion, Kooperation oder Offshoring (Outsourcing ins Ausland)  aufzeigen, 
als auch die Wettbewerbssituation besser einzuschätzen, um daraus Handlungsalternativen abzuleiten. 
In zahlreichen Studien tauchen in diesem Zusammenhang häufig sehr weitgefasste Begriffe auf wie „technology industry“ oder "IT-service-market" auf. 
Die Grenzen der Bandbreite der darin enthaltenen Beratungsleistungen sind darum besonders unscharf. Eine Differenzierung ist aufgrund der Komplexität der Zusammensetzung des Dienstleistungsspektrums daher nur im groben Umfang möglich.
Deswegen wurde darauf geachtet, dass die erfassten Unternehmen einen Mindestanteil  für ausschließlich beratende Tätigkeiten von mindestens 40\%  aufweisen, um in eine IT-Consulting äquivalente Kategorie zu fallen.
Es gibt zahlreiche Schlüsselfaktoren, die für die Einschätzung des Marktes in den jeweiligen Ländern bzw. Kontinenten aufschlussreich sind. 
Allerdings ist eine Erhebung sehr zeitaufwändig oder/und teuer. Es gibt zahlreiche Marktstudien von großen Marktforschungsunternehmen, wie z.B. Gartner, die man ab ca. 1000 €  erwerben kann, die aber nur Teilaspekte abdecken. 
Es gibt kaum kostenfreie umfassendere Studien zu dem gesamten Thema, dafür jedoch zahlreiche kostenpflichtige Studien zu hohen Preisen. Es lässt sich daher vermuten, dass es zumindest aufgrund des niedrigen Angebots an kostenlosen Studien, eine moderate Nachfrage nach kostenfreien Marktinformationen im Bereich des IT-Consulting gibt. Ob die kostenpflichtigen Studien den Informationsbedarf abdecken würden, gilt es daher zu prüfen und zu erwägen, ob eine Investition darin lohnt.

 Ziel des Kapitels ist es, sich dem  Thema zuerst theoretisch anzunähern und die Schlüsselfaktoren zu überlegen und deren konkrete Größen zu ermitteln. Anschließend erfolgt eine Untersuchung einzelner Teilaspekte.

\section{Teilaspekte}
Nachfolgend soll zuerst einmal begründet werden, welche Teilaspekte für eine Marktrecherche im IT-Consulting für eine Studie als besonders relevant erachtet werden.
Ein Großteil der Arbeit liegt deswegen vor allem in der Ermittlung des Informationsbedarfes um zu entscheiden, welche Kenngrößen als Faktoren überhaupt relevant sind.
Sichere Prognosen können daraus nicht in jedem Fall resultieren, da die gewählten Teilaspekte und deren Faktoren auf vernünftige Schätzungen beruhen und die Marktaspekte dynamische Faktoren eines komplexen und chaotischen System sind. Es lassen sich auf dieser Basis  jedoch vernünftige Schlussfolgerungen ziehen, die es wiederum zu verifizieren gilt. Statistische Verfahren bieten hier eine Möglichkeit, um zumindest eine hohe Wahrscheinlichkeit zu erreichen. 
Wie derart vage Informationen in eine Entscheidung einbezogen werden, bleibt außen vor. Es gibt jedoch zahlreiche Verfahren in der Entscheidungstheorie, welche unsichere Umstände in Managemententscheidungen einbeziehen. Letztendlich kommt es auf den Anwendungszweck der Information an, ob diese für eine Entscheidung relevant ist.
 
 \subsection{IT-Consulting Gesamtumsatz und Marktwachstum je Land}
 Diese Kennzahlen liefern wichtige Hinweise, wie sich die Branche weiter entwickeln wird und wie sehr sie schon entwickelt ist. 
 Diese Daten dienen dazu um Rückschlüsse auf die Nachfrage von IT-Consulting-Leistungen zu ziehen. 
 Dazu werden natürlich noch eine Reihe weiterer Daten benötigt, wie Gesamtmarktvolumen und Gesamtwirtschaftswachstum, um die Kennzahlen ins Verhältnis zu setzen.
  Außerdem muss darauf geachtet werden, welche Umstände die Kennzahlen verfälschen wie z.B. die Ländergröße, wodurch eher ein höherer Umsatz pro Land entsteht. 
  Es werden deswegen weitere Kennzahlen wie z.B. Fläche des Landes oder Einwohnerzahlen benötigt, um diese Faktoren ordnungsgemäß bewerten zu können.
 Der IT-Consulting-Umsatz und das Wachstum je Land ist im Verhältnis zu anderen Kenngrößen durchaus im vertretbaren Umfang anhand von öffentlichen Studien erfassbar.  
 Daher sollen diese Größen einzeln nach Ländern aufgelistet und erläutert werden. 
  Dabei ist zu berücksichtigen, dass auf die daraus resultierenden Anteile und Kennzahlen, dem Anspruch auf Korrektheit nicht gerecht werden kann. 
  Hierfür müssen diese Kennziffern noch mehrfach überprüft und Einflüsse die zur Verwässerung der Faktoren führen, herausgerechnet werden, um eine möglichst genaue Einschätzung zum Marktanteil und dem Wachstum zu treffen. Da dies jedoch den Rahmen der Arbeit sprengen würde, wird zunächst eine vereinfachte Betrachtung vorgenommen. Diese kann bei Bedarf als Ausgangsbasis  verwendet werden, um eine korrigierte Aussage zu treffen. 
  Für einen generellen Überblick und eine grobe Quantifizierung sind diese Daten essenziell. So können auf Basis dieser weitere Vermutungen angestellt und entsprechende Recherchen unternommen werden.
 Es wurde sich dazu entschieden diesen Teilaspekt für die Recherche auszuwählen. (siehe  \ref{subsubsec:Gesamtumsatz} \nameref{subsubsec:Gesamtumsatz} auf Seite \pageref{subsubsec:Gesamtumsatz})
 
\subsection{Gewinn- und Umsatzzahlen von Großunternehmen weltweit/Länder spezifisch}
Diese Daten sind gut zugänglich und liefern eine grobe Richtzahl über Umsatz und Erfolg in der Branche in dem jeweiligen Land. Hauptziel ist es eine Nachfrage aus den Kennziffern abzuleiten. So lassen sich anhand des Vergleiches der Umsatzzahlen und Gewinne der gleichen Unternehmen in verschiedenen Ländern Rückschlüsse auf mögliche Ursachen ziehen.
 Es muss hierbei jedoch berücksichtigt werden, ob die Großunternehmen nur Großprojekte übernehmen oder auch kleine bis mittlere Projekte. Sollte der hohe Umsatz maßgeblich durch Großprojekte generiert werden, muss dies nicht gleichermaßen kleine und mittlere Unternehmen gelten.
Des Weiteren dienen diese Daten dazu, die erfolgreichsten Unternehmen zu ermitteln, um diese anschließend zu beobachten und deren Wettbewerbsvorteile zu erkennen. Dieses Wissen liefert Hinweise auf Indikatoren, die zu dem Erfolg eines Unternehmens in dem jeweiligen Land beigetragen haben.
 So kann ein Unternehmen, welches z.B. auf SAP spezialisiert ist, zwar in Brasilien erfolgreich sein, jedoch in China trotz guter Marktlage nicht. Ursachen liegen in diesem Fall nicht im Markt, sondern z.B. in der strategischen Ausrichtung der IT in chinesischen Unternehmen (z.B. Kostendruck, kurzsichtige Denkweise) oder weil es sehr starke Konkurrenten gibt.
Signifikante Unterschieden bieten Potential um Vermutungen aufzustellen und diese weiter zu untersuchen oder um allgemeine Unterschiede im Erfolgspotential zwischen Ländern abzuleiten. 

 \subsection{Firmengrößen im IT-Consulting}
Ein wichtiger Einflussfaktor um die Konkurrenzsituation festzustellen, ist die Beschaffenheit des Marktes nach Unternehmensgrößen.
 Es stellt sich die Frage, ob eher Großkonzerne, mittelständische und kleine  Unternehmen oder gar interne Mitarbeiter für die strategische und ganzheitliche Gestaltung der IT beauftragt werden.
Es kann zum einen die durchschnittliche Firmengröße eines IT-Consulting Unternehmens herangezogen werden  und zum anderen die prozentualen Anteile der einzelnen Sektoren.  Es gibt Länder, die eher einen breiten Mittelstand haben oder eher von Großkonzernen dominiert werden. 
Ziel einer solchen Analyse ist es, Häufungen oder Bedarfe in einzelnen Sektoren festzustellen. 
Beispielsweise ein mittelständisches Unternehmen könnte, wenn es nur einen geringen Anteil kleiner und mittelständischer Unternehmen gibt, von einem erhöhten Bedarf in der IT-Beratung mit kleiner bis mittlerer Projektgröße profitieren, in so fern die Großunternehmen nur an großen Projekten interessiert sind.

Die Möglichkeiten zur Ermittlung der Marktbeschaffenheit bestehen aus  folgenden Schritten:
\begin{itemize}
\item Statistische Stichproben / Befragungen zur Ermittlung der Anteile
\item Analyse Dienstleistungsangebote der IT-Consulting-Unternehmen in dem jeweiligen Sektor 
\item Berechnung der Marktanteile der größten Unternehmen und Bewertung des Restwertes zum Vergleich mit der Stichprobe
\item Beschaffung von Studien oder Beauftragung eines Marktforschungsunternehmen, falls die Stichproben nicht ausreichen
\end{itemize}
Die resultierenden Informationen zur Beschaffenheit geben wertvolle Hinweise über das Angebot in der Branche des IT-Consulting. 
So können von diesem Wissensstandpunkt aus die konkreten Angebote analysiert werden und ggf. Chancen abgeleitet werden. 
Beispielsweise können große Unternehmen aufgrund ihrer Kosteneffizienz möglicherweise bessere Qualität anbieten als ein breiter Mittelstand, gleichzeitig die Nachfrage nach mittelständischen Unternehmen aufgrund günstigerer Preise höher sein. 
Eine tiefgehende Analyse ist jedoch sehr komplex und würde Stoff für eine eigenständige Studie liefern und soll deswegen kein weiterer Gegenstand sein.

 \subsection{Politik / Rechtslage}
Die politische und rechtliche Lage ist ein wichtiger Faktor für die Wahl eines Standortes oder die Zusammenarbeit mit einem Land. Gleichzeitig ist es ein Teilaspekt, der die Entwicklung der Branche beeinflusst.
Besonderen Einfluss haben die folgenden Faktoren auf die IT-Consulting Branche:
\begin{itemize} 
\item {generelle staatliche Subventionen / Investitionen in die IT}

 Staatliche Förderprogramme und Subventionen für IT-Entwicklungen stellen eine wichtige Finanzierungsmethode für Entwicklungsprojekte oder Unternehmen, deren Existenz bedroht ist, dar. Diese Subventionen bestehen beispielsweise aus zinsgünstigen Darlehen oder Fördermitteln, die nicht zurückgezahlt werden müssen. Die unterschiedlichen Arten von Fördermitteln müssen daher klassifiziert und entsprechend eingeordnet werden
 Solche finanziellen Unterstützungen treiben natürlich die Forschung und den allgemeinen Wissenstransfer voran. Insbesondere das IT-Consulting ist eine reine Wissensbranche. Wenn die IT-Entwicklung von Unternehmen profitiert, profitiert auch das IT-Consulting von den neuen Erkenntnissen. 
 Staatliche Finanzierungsformen können als Anreiz für Unternehmensgründungen dienen und damit die Innovativität der IT-Landschaft fördern. Neue Technologien und Erkenntnisse führen zu neuen Handlungsalternativen. Das Wachstum in der IT-Branche erhöht wiederum die Komplexität von Entscheidungsprozessen und Nachfrage nach Beratungsleistungen.
Ein Wissen über Staatssubventionen kann daher hilfreich sein, um das Entwicklungspotential und das Maß an finanzieller Stabilität für Unternehmens-Neugründungen, welche durch Förderprogramme erhöht werden kann, einzuschätzen.
 \\
\item  {Handelsrecht / Arbeitsrecht}

 Das Vertrags-und Arbeitsrecht hat Einfluss auf das Outsourcing oder die Kooperationen mit einem ausländischen Unternehmen. 
 So sind in jedem Land bestimmte Handels und Arbeitsgesetze zu beachten, die eine reibungslose Unternehmens-Kooperation oder ein rechtmäßiges Arbeitsverhältnis gewährleisten. 
Für das Unternehmen steht vor allem die Frage nach der Haftung und Behandlung von Mängeln im Vordergrund. 
Eine Einordnung nach Risiken und Chancen ist hier folglich notwendig, um das Land zu beurteilen. 
 \\
\item {Steuerrecht}

 Das Steuerrecht ist vor allem für die Standortwahl ausschlaggebend. 
 So sind möglicherweise bestimmte Besteuerungsvorschriften mit in die strategische Standortauswahl einzubeziehen. So gilt es abzuwägen, ob die Gewinnerwartungen nach Steuern höher sind, als in einem anderen Land. Zahlreiche Unternehmen suchen sich Ihren Hauptsitz daher nach den für Sie günstigen Steuervorteilen aus.
 Allerdings gilt es verschiedene Punkte zu analysieren wie:
 - Höhe der Mehrwertsteuer
 - Legalität bei Dienstleistungsvertrieb und Aufenthalt in einem anderen Land
 - Höhe der Einkommensteuern
 - Höhe der Gewerbesteuer/Grundsteuer
 - Spezielle Sonderregelungen
  \\
\item {Datenschutz/Urheberrecht}

  Der Datenschutz ist ausschlaggebend für den Austausch von sensiblen Unternehmensdaten. 
  Die Gefahr von Produktpiraterie oder der Weitergabe von sensiblen Daten kann ein Risiko für Kooperation mit Partnern, Expansion oder dem Outsourcing sein. Ursachen liegen hier in zu schwachen oder gar nicht vorhandenen Gesetzen für das Urheberrecht oder dem Datenschutz.
  
  \end{itemize}

\section{Analyse ausgewählter Teilaspekte}
Aufgrund des hohen Rechercheumfangs im Bereich Markt wurde sich auf den Teilaspekt "IT-Consulting Gesamtumsatz und Marktwachstum je Land" beschränkt. Dafür wurde besonders auf Vollständigkeit der Daten für einen Vergleich geachtet. 
\subsection*{IT-Consulting Gesamtumsatz und Marktwachstum je Land}
\label{subsubsec:Gesamtumsatz}

\begin{itemize} 
\item {Global}

Insgesamt wurden 2010 laut Gartner 574,94 Milliarden Euro weltweit in der IT-Consulting-Branche umgesetzt. \cite{itConsultingGlobal} Das durchschnittliche globale jährliche Wachstum beträgt 2,6\% zwischen 2007 und 2011.\cite{globalGartner}

\begin{figure}
  \centering
  \includegraphics[width=0.8\textwidth]{images/global_revenue_share.jpg} 
  \caption{Anteile am IT-Consulting Weltmarkt} \label{fig:weltmarkt} 
\end{figure}


\item {Deutschland}

Das Wachstum des deutschen IT-Consulting ist mit 8,4\% sehr hoch im Verhältnis des Wirtschaftswachstums von 3\% in 2011.\cite{statGer2} Dabei haben die Top 25 IT-Consulting- Unternehmen sogar noch ein größeres Wachstum mit Spitzen bis zu 10\%. \cite[6]{topITB} Dies sind durchaus überdurchschnittliche Wachstumsraten, welche international mit aufsteigenden Ökonomien wie Indien, Russland oder China sehr gut mithalten können.  (siehe  \ref{table:umsaetze} \nameref{table:umsaetze} auf Seite \pageref{table:umsaetze}) Der Gesamtumsatz des IT-Consulting beträgt 29,4 Milliarden Euro.  Das sind 1,12 Prozent des gesamten bereinigten Bruttoeinlandproduktes. \cite{statGer} Um die Dichte des IT-Consulting-Marktes einschätzen zu können macht es Sinn den Gesamtumsatz auf die Ländergröße zu beziehen. 
Deutschland erzielt 8,24 Milliarden Euro pro 100000 km² ab. Dies ist im Vergleich mit den anderen Ländern eine enorm hohe Dichte und bietet dadurch eher Wegersparnisse und Kommunikationsvorteile, die sich strategisch günstig auswirken können. 

\item {USA}

Die USA zweifellos einen gigantischen Marktanteil an IT-Services. Mit 244 Milliarden Euro, was 43\% des weltweiten IT-Service Markt einnimmt, ist es mit Abstand das umsatzstärkste Land der Welt im Bereich IT-Services. \cite{ibisUSA} Das Wachstum stagniert allerdings mit 2,2\%. Im Verhältnis zum Wachstum des bereinigten Bruttoninlandsproduktes von 1,8\% wächst es nur wenig mehr. \cite{statUSA} Die Dichte des IT-Consulting-Marktes beträgt 2,48 Milliarden Euro pro 100000 km². Im Verhältnis zu anderen Ländern wie z.B. Indien oder Russland ist dies eine sehr hohe Dichte, nur Deutschland schneidet noch deutlich besser ab.

Ein Großteil der Umsätze in den USA entsteht unter anderem dadurch, dass Wertschöpfung durch IT-Services, die im Ausland durch Offshoring entstehen, hinzugerechnet werden. Diese Offshoring-Länder sind daher einer der Schlüsselfaktoren für die hohen Umsätze in den USA. 

\item {China}

China hat mit 72,6 Milliarden Euro den zweitgrößten Marktanteil der Welt. Es hat zwar noch weniger als ein Drittel gegenüber den USA, jedoch verzeichnet es mit 6,8\% in 2011 überdurchschnittliche Wachstumsraten im Bereich IT-Services und hat damit das dreifache Wachstum des Konkurrenten USA.  Im Verhältnis zum Wachstum des Bruttoinlandsproduktes mit 9,2\% in 2011 ist das Wachstum jedoch verhältnismäßig wenig. Dieses Verhältnis deutet stark daraufhin, dass die strategische Ausrichtung der IT und deren Prozessen noch nicht gleichermaßen Beachtung geschenkt wird, wie anderen Dienstleistungen und Produkten. Mögliche Ursachen könnten Fachkräftemängel oder Kostendruck sein. Es gilt daher weiter zu untersuchen, welche Ursachen das verhältnismäßig schwächere Wachstum hat. 
Die Wachstumsraten im Bereich IT-Services sind in alle Ländern über dem des BIP. Warum dies in China so weit nach unten abweicht ist, gilt es daher weiterhin zu untersuchen und dafür Ursachen zu finden.
Die Dichte ist im Verhältnis zu Deutschland oder den USA auch deutlich schwächer. Hier schneidet China mit einer IT-Service-Dichte von 0,74 Milliarden pro 100000 km² ab. Da China sehr weitläufig und nicht vollständig industrialisiert ist, ist diese Größe wenig aussagekräftig und es besteht weiterer Untersuchungsbedarf, ob diese Größe wirklich auf einen schwächer entwickelten IT-Consulting-Markt hindeutet oder anderen demografischen Umständen geschuldet ist. \cite{ibisChina}

\item {Russland}

Der russische Markt im Bereich IT-Services ist mit 14,3 Milliarden recht klein wenn man es auf die Größe des Landes bezieht. Es gibt vor allem viel System- und hardwarenahe Entwicklung. Dienstleistungen im IT-Sektor wachsen jedoch in zunehmenden Maße und konnten 2011 15,3\% Branchen-Wachstum erreichen. Dies ist sowohl im Vergleich zum internationalen Markt als auch im Verhältnis zum Wachstum des Bruttoinlandsproduktes mit 4,3 \% in 2011 weit überdurchschnittlich. \cite{statRus2} Aufgrund der relativ kostengünstigen Entwicklungskosten für Software, insbesondere hardwarenahe Entwicklung und Systemengineering wird Russland vor allem in Europa zunehmend als „Offshoring Land“ attraktiv. (Wirtschaftsinformatik und Management 12/2013, Offshoring Land Russland)
Die Dichte des IT-Consulting fällt mit 84 Millionen Euro pro 100 000 km² sehr klein aus. Dabei ist zu berücksichtigen, dass ein Großteil von Russland gar nicht oder nur schwach bewirtschaftet ist. \cite{statRus}


\item {Afrika}

Afrika hat einen sehr niedrigen Anteil am Markt mit 1,4 Milliarden. \cite{statAfr} Statistiken zum Wachstum konnten leider nicht gefunden werden. Die großen Technologie-Beratungs-Konzerne haben sich in verschiedenen Teilen Afrika als Beratungen etabliert, die auch den IT-Consulting Markt abdecken. Der Trend geht jedoch laut Experten immer mehr dahin, dass neue inländische IT-Service-Provider auf dem Markt konkurrieren und die US-Konzerne ablösen. Afrika hat 2011 mit 5\% ein durchaus gutes Wirtschaftswachstum zu verzeichnen, kann aber nicht mit anderen aufstrebenden Ökonomien mithalten, insbesondere nicht im IT-Umfeld. \cite{statAfr2}
Ursachen liegen hier nicht zuletzt in der Stromversorgung. Denn 80\% der Dörfer in Afrika sind immer noch ohne Stromversorgung, weil Energieanbietern die Vernetzung der Dörfer zu teuer ist.  \cite{dieZeit}
Die Situation des Marktes im gesamten Kontinent ist jedoch sehr vielschichtig. Eine Analyse des Marktes in sehr komplex, da in Afrika sehr viele demografische und infrastrukturelle Umbrüche stattfinden. Daher ist hier eine Betrachtung der Umsätze wenig aussagekräftig für eine klare Einschätzung. 
Um eine besser Bewertung zu ermöglichen, ist eine Unterteilung von Afrika notwendig. Des weiteren sind die Kennzahlen, aufgrund der hohen Marktdynamik, aus nur einem Jahr nicht ausreichend aussagekräftig. Deswegen ist es sinnvoll die Umsätze und Wachstumsraten, noch ein einem größeren Zeitraum zu betrachten.
Es besteht daher hier weiterer Forschungsbedarf, um die Kennzahlen in einem differenzierteren Kontext zu stellen.
Die Werte eignen sich daher mehr für den Vergleich der IST-Situation mit anderen Ländern.


\item {Brasilien}

Brasilien hat im Jahre 2011 mit einem Branchenumsatz von 69,6 Milliarden, den drittgrößten Anteil am globalen IT-Consulting Markt erzielt. Das Wachstum über die Jahre von 2008-2011 beträgt 61\%. \cite{statBras2} Trotz seiner flächenmäßigen Größe weist es mit 0,82 Milliarden pro 100 000 km eine verhältnismäßig hohe Dichte auf. Das Wachstum in 2011 in der IT-Services-Branche beträgt hier 4,9\% und ist damit weitaus höher als das Gesamtwirtschaftswachstum mit 2,7\%. Dabei beträgt der Anteil am Bruttoinlandsprodukt allein 4,5\%. Dies zeigt welchen großen Stellenwert der Markt für IT-Services in Brasilien einnimmt. Experten prognostizieren weiterhin einen rasanten Anstieg des Wachstums, insbesondere im Bereich BPO, welche laut Schätzungen 85\% am Gesamtmarktanteil einnimmt.\cite{statBras}

\item {Zusammenfassung}

Alle Werte beziehen sich auf die Jahre von 2011 bis 2012. Die entsprechenden Quellen sind in den Länderabschnitten zu finden.


\begin{table}

\caption{Übersicht Umsatz und Umsatzwachstum von IT-Services im Verhältnis zum BIP  (2010)}
\begin{tabular}{|p{2.6cm}|p{1.5cm}|p{2cm}|p{1.5cm}|p{1.5cm}|p{1.7cm}|}
 \hline
  \textbf{Land} & \textbf{Umsatz in Mrd. €} & \textbf{IT-Consulting Wachstum} & \textbf{BIP Wachstum} & \textbf{Welt-Markt-Anteil} & \textbf{Umsatz-Dichte in Mrd. € pro 100 000 km²} \\
  \hline
    
    1. USA  & 244,33  &2,2\%  & 1,8\% & 43\% & 2,48  \\
    2. China & 72,50 & 6,8\%  & 9,2\% & 13\% & 0,74 \\
    3. Brasilien & 69,60 & 4,9\%  & 2,7\% & 12\% & 0,817 \\
    4. Deutschland & 29,4 & 8,5\%  & 3\% & 5\% & 8,24 \\
    5. Indien & 25,45  & 11.2\%  & 7.9\% & 4\% & 0,77  \\
    6. Russland & 14,3  & 15.4\%  & 4,3\% & 2\% & 0,084  \\
    7. Afrika & 1.4  & n/a  & 5\% & 0.5\% & 0,0046 \\
 \hline
\end{tabular}
\label{table:umsaetze} 
\end{table}

\end{itemize}




\section{Arbeitskultur}
	\subsection{Einleitung Arbeitskultur}
Arbeitskultur ist ....
Arbeitskultur ist eine Teilmenge der Kultur(Sitten,Bräuche) einer Nation. Sie gehört zum Beratungsprozess und spielt dabei nicht die unwesentlichste Rolle. IT-Consultans kennen innerhalb der wenigsten Zeit sehr viele Firmen und deren Mitarbeiter kennen. Berater arbeiten öfter durch Iteration mit Menschen:\\
1)aus unterschiedlichen Unternehmensebenen, angefangen von normalen Mitarbeiter(Interaktion mit Beratern während den Schulungsmaßnahmen bei IT-Neueinführung) bis zu Top-Managementebenen(Interaktion mit Beratern während der strategischen Fragen wie Planung von Anwendungssoftware, Analyse von Geschäftsprozessen usw.)\\

2)aus verschiedenen Branchen wie Finanzdienstleistung,Fahrzeugbau, Großhandel usw.
\begin{figure}[htp]
\centering
\includegraphics[width=10 cm]{./images/Auft_U_Beratung}
\caption{Aufteilung Unternehmensberatungen nach Branchen}
\label{fig:AufteilungUnternehmensberatung}
\end{figure}

3) aus unterschiedlichen Länder mit jeweils eigenartigen Kulturen.\\
 In diesem Punkt werden 2 Standardfälle erläutert um die Bedeutung der Arbeitskultur im Beratungsprozess zu zeigen.\\
 a) Das erste Fall ist ein IT-Consulting Unternehmen mit Beratern die aus unterschiedlichen Ländern kommen, die unterschiedliche Sprache sprechen und sich kulturell enorm unterscheiden.Diese Berater arbeiten zielgerichtet und ständig im Team. In diesem Fall wird dem Author dieser Arbeit sehr interessant inwieweit sich kulturelle Unterschiede auf das gemeinsame Ziel des Beratungsprozesses bei der Softwareeinführung auswirken können. Auch interessant ist hier wie die IT-Berater aus unterschiedlichen Länder mit Kunden aus Deutschland umgehen, ob die kulturelle Unterschiede einen Einfluss auf Kundenbeziehungen haben oder nicht. \\
 b) Das 2. Fall bezieht sich auf ein deutsches Unternehmen, das sich international agiert und Kunden aus unterschiedlichen Länder betreut. In diesem Fall müssen sich deutsche Mitarbeiter auf unterschiedliche Arbeitskulturen anpassen. Denn ein Meeting während des Mittagsessen in Japan ist widersinnig und wirkt unseriös, in USA dagegen ist es nicht ungewöhnlich, dass beim Essen wichtige Entscheidungen kollaborativ getroffen werden.  \\
Wegen der zeitlichen sowie thematischen Begrenzung liegt der Autor dieser Arbeit den Fokus nicht auf die Differenzierung dieser zwei Fälle sowie kulturelle Unterschiede der Berater, sondern nur auf die unterschiedliche Arbeitskulturaspekte, die für den Beratungsprozess ausschlaggebend sind. Teilaspekte der Arbeitskultur, die den Autoren dieser Arbeit interessant erscheinen, werden in folgenden Kapiteln vorgestellt und verglichen. In diesem Sinne werden diese 2. Fälle nicht weiterhin detaillierter behandelt.

\subsection{Einleitung in das Wesen des IT-Consultings}	

IT-Consulting ist eine wichtige Art des Consultings in IT-Fragen eines Unternehmens. Das Wesen des Consultings besteht im Allgemeinen darin, Unternehmen bei der Neustrukturierung des Anwendungslandschaften oder bei der Pflegung der bestehenden Informationssysteme zu unterstützen. Während des gesamten Beratungsprozesses bleibt Berater als externe Experte solange im Unternehmen bis die Probleme,die er mit seinem technischen Fachwissen zu lösen hat,nicht mehr existieren oder selbständig von den Mitarbeitern des Unternehmens gelöst werden können.
Um den Beratungsprozess zu verdeutlichen wird jetzt ein Beispielprozess aus der Praxis der IT-Beratung beschrieben. Ein Online-Handelsnternehmen möchte ein BI Standardsoftware einführen um die Daten für Analysezwecke aus dem ERP-System zu laden,um die potentiellen Kündiger zu vermeiden oder neue Kunden zu gewinnen.Am Anfang jedes Prozesses muss dem Berater die Organisationsstruktur des Unternehmens klar sein, um eine passende Lösung zu finden. Im Beratungsprozess gibt es eine Standardsoftware um IT-Problem zu lösen. Es gibt aber keine Standardlösung die für alle Unternehmensstrukturen passend ist,weil die Unternehmensstrukturen sehr unterschiedlich sind. Man beginnt die Verhandlung zwischen Unternehmensführung und den Beratern, die den Auftrag bekommen, indem man Vertrag abschließt. Danach beginnt die Analysephase. Hier wird die Unternehmensstruktur des Online-Handelsunternehmen auseinandergenommen, bis man erkennt wo die Software eingesetzt wird, Stellen wo die Reibungen entstehen werden,welche Ressourcen stehen zur Verfügung und welches Informationssystem sich am besten dafür eignet.
Es muss immer ein Feedback zwischen dem Berater und Unternehmensführer möglich sein.
nach der Analysephase beginnt die Umsetzungsphase indem eine neue IT-Architektur aufgebaut wird oder die vorhandene ergänzt wird.Im unseren Beispiel wird die ERP-Lösung mit dem BI Lösung erweitert, die vorhandene Architektur bleibt erhalten. In dieser Phase können auch die andere Berater aufgerufen werden, falls es viele komplizierte Realisierungsmaßnahmen gibt.
Nachdem die Informationssystem erfolgreich installiert ist, beginnen die Schulungsmaßnahmen, damit die Mitarbeiter des Unternehmen in der Lage sind mit diesem System umgehen zu können. Zum Schluss kommt die Wartungsphase und Intensität der Beratungsdienstleistung nimmt langsam ab. 
	\subsection{Bedeutung der Arbeitskultur für IT-Consulting}
In wie weit ist es wichtig Arbeitskultur für den Beratungsprozess zu betrachten? Anhand vom unseren Beispiel ist es zu erkennen dass die Berater in jeder Phase der Softwareeinführung mit den Unternehmensvertretern kommunizieren sollen. Es ist wichtig,dass die Berater genug technisches Know-how mitbringen, noch wichtiger sind die Soft Skills, die für erfolgreiche Geschäftsbeziehungen entscheidend sind.``IT Business is People's Business. Diese Leitlinie impliziert, dass der Erfolg von IT-Projekten maßgeblich von der Kompetenz des Beraters abhängt.´´
%(Quelle: http://www.it-production.com/index.php?seite=einzel_artikel_ansicht&id=26189)
 Welche Social Skills des IT-Beraters sind für Deutscher als obligatorisch herausgestuft? Sind diese persönlichen Eigenschaften auch für die anderen Nationen von der Bedeutung? Unternehmensführung und IT-Berater müssen bei der Lösung des Problems einig werden. Der Berater muss Unternehmen für seine vorgeschlagene Lösung überzeugen. Muss man, um dies zu realisieren, nur eine gute Software anbieten und als vertrauenswürdiges Unternehmen am Markt agieren oder reichen diese Bedingungen beispielsweise in Indien nicht aus ,weil der Berater aus anderer Kaste ist.Denn die Kastenzugehörigkeit hat in Indien bis heute kulturelle und soziale Auswirkungen auf viele Lebensbereiche. % (Quelle: http://www2.klett.de/sixcms/list.php?page=geo_infothek&miniinfothek=&node=Indien&article=Infoblatt+Kastensystem+in+Indien)
 Für diese Arbeit ist wichtig zu wissen wie die Arbeitskultur in ausgewählten Länder sich unterscheidet und in wie weit diese den Beratungsprozess beeinflussen kann.
 
%Sind überhaupt Softs Skills entscheidend für einige Länder oder spielt eher die technische Ausrüstung des Beraters bedeutsame Rolle. Schwerpunkte:\\ 1)Für die Autoren dieser Arbeit ist es aus persönlichen Interessen wichtig zu wissen wie man Unternehmen aus anderen Ländern berät und indem man als Berater ins Ausland geschickt wird.\\2)Wie verläuft der Beratungsprozess in einigen Länder die bedeutungsvoll und potenzialreich für IT-Consulting sind.



In den folgenden Kapiteln wird Arbeitskultur von ausgewählten Länder(Russland, USA, Deutschland usw.) untersucht und zum Schluss werden einige interessante Fakten verglichen und diskutiert. 

\subsection{Teilaspekte der Arbeitskultur und ausgewählte Länder}
Hier werden Aspekte aufgelistet die für Arbeitskultur von der Bedeutung sind (Arbeitsplatz, Mitarbeiterverhältnisse,Hierarchien,Organisation,Lebensunmstände etc).
Autoren dieser Arbeit haben sich einige Teilaspekte der Arbeitskultur sowie die zugehörigen Länder durch Brainstorming überlegt. Dazu wird übersichtshalber eine Matrix aufgestellt. Felder dieser Tabelle bleiben zuerst leer und nach dem die einzelne Aspekte von den Ländern recherchiert und vorgestellt werden, wird die Matrix noch mal mit den ausgearbeiteten Feldern ausgefüllt. damit kann man auch die Rechercheergebnisse testen, ob sie erfolgreich waren oder nicht. \\
\\
\begin{table}[htp]
\begin{tabular}{|c|c|c|c|c|c|c|}
\hline  Aspekt/Land& Deutschland & USA & Russland & Japan & Indien & Brasilien \\ 
\hline Hierarchien  & ? & ? & ? & ? & ? & ? \\ 
\hline  Kundenverhältnisse& ? & ? & ? & ? & ? & ? \\ 
\hline  spezielle Rechtslage& ? & ? & ? & ? & ? & ? \\ 
\hline  Grad des intuitiven Handelns& ? & ? & ? & ? & ? & ? \\ 
\hline  Kritikfähigkeit& ? & ? & ? & ? & ? & ? \\ 
\hline  Tagesrythmus& ? & ? & ? & ? & ? & ? \\ 
\hline  Organisation& ? & ? & ? & ? & ? & ? \\ 
\hline  Lebensumstände& ? & ? & ? & ? & ? & ? \\ 
\hline  Zeitmanagement& ? & ? & ? & ? & ? & ? \\ 
\hline  Work-Life-Balance& ? & ? & ? & ? & ? & ? \\ 
\hline 
\end{tabular} 
\caption{Matrix der Arbeitskultur, Quelle: eigene Darstellung xD}
\end{table}


	\subsection{Russland}
	% was mache ich mit den Quellen in original Sprache?-genau so wie de 
	Russland ist ein Wachstumsmarkt mit Zukunft. Heute ist der damals geschützter russischer Markt offen für Exporte und Investitionen aus Deutschland.Dies gilt sowohl für IT-Beratungs Unternehmen, die ihre Softwareprodukte in Russland integrieren auch für russische Manager, die bei der Informationstechnologie auf westliches Know-how setzen.\\
	%(Quelle: http://www.it-production.com/index.php?seite=einzel_artikel_ansicht&id=26189)
	Da der Markt für IT-Beratung neu ist, muss man als IT-Berater aus Westen ganz viele Entscheidungen intuitiv treffen. Hier weden natürlich die Soft Skills gefragt. Technische Fähigkeiten, funktionales Wissen und Branchen-Know-how sind selbstverständlich vorausgesetzt. Sonst wären die höheren Tarife für westlichen Berater ungerechtfertigt. Mit anderen Wörtern müssen die deutschen Beratern mit ihrem Wissen auf einer Stufe höher stehen als russischen Kollegen, damit sie für russische Softwareprojekte eingesetzt werden können. 
	Der russischer Senior-Consultant aus Moskau verdient im Mittel 3845 € im Jahr. Das ist für russische Verhältnisse relativ hoher Gehalt. Doch gibt es in Russland sehr starke regionale Gehaltsunterschiede. Aus dem unten stehenden Diagramm kann man den Unterschied des monatlichen 
	Gehalts für SAP-Berater ermitteln. Im Großen und Ganzen verdient man in beiden Metropolen Moskau und Sankt-Petersburg ca. das doppelte wie in anderen Großstädten wie Rostov, Wolgograd oder Omsk.
	\\
\begin{figure}[htp]
\centering
\includegraphics[width=0.7\linewidth]{./images/SAP-Berater_Gehalt_RU}
\caption{SAP-Berater Gehälter in Russland}
\label{fig:SAP-Berater_Gehalt_RU}
\end{figure}

	
	%Quelle:http://www.tadviser.ru/index.php/\CYRS\cyrt\cyra\cyrt\cyrsftsn\cyrya:\CYRR\cyrery\cyrn\cyro\cyrk_\cyrt\cyrr\cyru\cyrd\cyra_\cyrv_\CYRR\cyro\cyrs\cyrs\cyri\cyri_\CYRI\CYRT_\cyri_\cyrt\cyre\cyrl\cyre\cyrk\cyro\cyrm
	


	Senior IT-Berater aus Deutschland verdient im Mittel 6.250 € im Monat. Die deutschen Berater, die nach Ausland geschickt werden, haben noch höheren Gehaltstarife. %(Quelle: http://www.computerwoche.de/a/sap-berater-90-000-euro-nach-fuenf-jahren,2535266)
	Der Junior Berater im IT Umfeld ohne Projekterfahrung  verdient in Russland ca. 1200 € monatlich.
	Sein deutscher Kollege mit den gleichen Qualifikation und Erfahrung verdient ca. drei mal so viel (3750 €).
	%Quelle:http://www.computerwoche.de/a/sap-berater-90-000-euro-nach-fuenf-jahren,2535266
	\\
	\\
	´´Der Zerfall der Sowjetunion und die Reformen im wirtschaftlichen und sozialen Gefüge Russlands haben einen erheblichen Einfluss auf die Arbeitskultur in gegenwärtigen russischen Organisationen´´. Deswegen überlappen sich die kulturelle mit reformbedingten Faktoren der Arbeitskultur. Autor dieser Arbeit möchte eher auf Beschreibung von Teilaspekten  eingehen und nicht auf die Erklärungen der Tatsache. Ursachen.\\%Quelle http://joconsult.netzmerk.com/pup/prozess-de.pdf
	
	%	\textbf{Aspekt Arbeitsplatz}\\``Der Arbeitsplatz ist für viele russische Arbeitnehmer nicht nur der Ort, an dem das Einkommen erarbeitet wird, er hat auch eine große soziale Bedeutung: Familienangehörige von Kollegen kennen sich untereinander, wenn der Kindergarten geschlossen hat, bringen Frauen ihre Kinder mit zur Arbeit etc. Eine Abgrenzung zwischen Berufstätigkeit und Arbeitsleben, wie sie im westeuropäischen Sinne üblich ist, wird nicht vorgenommen.``-> Wie kann die Aussage den Beratungsprozess beeinflussen?\\
	\textbf{Team und Organisation: Russischer Arbeitskollektiv gegen westlichen Team}\\
	``Das Arbeitskollektiv wurde in der sowjetischen 
	Epoche als das zentrale soziale Handlungsfeld propagiert und die Geschlossenheit der Gruppe ist wichtiger als die Selbstverwirklichung des Einzelnen Gruppenmitglieder.`` Gruppeninterne Konflikte wurden deshalb vermieden oder nicht diskutiert. Der Unterschied gegen dem westlichen Team besteht darin, dass russischer Kollektiv eine dauerhafte Einrichtung ist klar zugewiesene Leitungskompetenzen, die vom Vorgesetzten ausgeübt werden, während das Team nur für die Dauer eines bestimmten Projektes eingerichtet wird und sich durch die Gleichberechtigung aller Teammitglieder auszeichnet.Quelle:
	So ein Kollektiv für Beratungszwecke ist demzufolge nicht flexibel und ist zu stark weisungsgebunden. Die Aufgaben im Kollektiv werden vom Vorgesetzten vorgeschrieben, im unser Fall von einem Projektleiter oder einem Manager. Solche Führungspersonen sind im IT-Beratungsfall an einen Office gebunden und sind immer im Office während die Berater immer unterwegs bei den Kunden sind. Daher müssen die Entscheidungen immer intuitiv und unabhängig von dem Vorgesetzten getroffen werden. Das ist eine Widerspiegelung dem Prinzip des russischen Kollektivs. ``Russische Organisationen zeichnen sich durch eine Konzentration von Macht auf die Führungskräfte aus. Ohne den ``Natschalnik`` werden keine Entscheidungen getroffen.`` %Quelle http://joconsult.netzmerk.com/pup/prozess-de.pdf
	Verlagerung von Entscheidungen auf die Mitarbeiter wird in Russland selten stattfinden,deswegen werden die Mitarbeiter von der Verantwortung befreit und übernehmen oft nur Anweisungsfunktionen. Für  den Beratungsprozess ist diese Tatsache ein reisen Minuspunkt, weil die Berater das interdisziplinäres Wissen besitzen und  den vollen Handlungsspielraum in der IT-Beratungsszene brauchen.
	Zu erwähnen wäre noch, dass die jungen Menschen von solcher Stereotypen weiter entfernt sind als die ältere ``sowjetische`` Generation. \\

	\textbf{Personalauswahl und Gesetze}\\
	 Eine weitere wichtige Besonderheit ist die Personalauswahl. Häufig erfolgt die Auswahl von neuen 
	 Mitarbeitern nicht nach Kriterien der fachlichen Kompetenz. Oft werden Arbeitsplätze unter Verwandten und 
	 Freunden vergeben. Es existieren fast keine etablierten Mechanismen von Angebot und Nachfrage auf dem 
	 Arbeitsmarkt. Vakanzen werden häufig nicht an den fachlich geeignetsten Bewerber vergeben, sondern an 
	 ``unseren Mann``(nash celovek).\\
	 Ein weiteres für Russland typisches Merkmal ist, dass Gesetze, Bestimmungen und 
	 Regelungen keinen eindeutig verbindlichen Charakter haben. In Abhängigkeit von der 
	 Situation und den involvierten Personen, können Regeln oder Gesetze bewusst 
	 unberücksichtigt bleiben. Wie sich jedoch diese Abstufung darstellt ist nicht vorhersagbar. Das liegt auch daran, dass das russische Volk und die russischen Behörde sich einander nicht zutrauen.
	 %Quelle http://joconsult.netzmerk.com/pup/prozess-de.pdf
	 
	 \textbf{Arbeitszeit und Urlaub}\\
	 Die gesetzliche Wochenarbeitszeit in Russland beträgt 40 Stunden. Doch in meisten Fällen wird diese Grenze total überschritten. Die IT-Spezialisten arbeiten zwischen 10 und 11 Stunden am Tag in einem 5 Tage-Rhythmus. Oft wird auch 6- Tage- Woche praktiziert. Die deutschen Berater arbeiten ca. 10 Stunden  im 4 Tage-Rhythmus und sind direkt beim Kunde vor Ort. Am Freitag gibt es entweder Zeit für eigene Weiterbildung im Home-Office oder ein lokales Meeting in der Firma bis nachmittags.
	 In vielen Tarifverträgen in Deutschland beträgt der Jahresurlaub 30 Arbeitstage. In Russland sind es dagegen nur 24 Tage. Der Arbeitstag beginnt bei russischen nicht produzierenden Firmen um 9 oder 10 Uhr. Wenn ein IT-Berater um 10 Uhr mit seiner Arbeit beginnt, dann ist er um 20-21 Uhr zu hause. Natürlich bleiben viele Sachen zu hause liegen und es bleibt sehr wenig Zeit für privates Leben.  %Quelle: http://de.rusbiznews.com/about/PervyirazvRossii 
	 \\
	 
	 \textbf{Reisen und Pünktlichkeit}\\
	 IT-Berater Beruf ist eine Tätigkeit, die mit höheren Reisebereitschaft verbunden ist. In Deutschland sind die Berater oft mit Autos unterwegs. Von einem deutschen Großstadt bis zum anderen braucht man beispielsweise 4-5 Stunden.In Russland gibt es 2 grundsätzliche Transportprobleme mit dem Consulting-Hintergrund, die dem Autor auf den Ersten Blick erscheinen: Staus in Moskau und große Entfernungen zwischen den russischen Städten.Das Land ist nahezu unendlich groß und weit(es umfasst 11 Zeitzonen).
	 Zwischen Moskau und Nowosibirsk sind es ca 4 Stunden nur Flugzeit und plus 3 Stunden Zeitunterschied. Wenn ein Berater aus Moskau seinen Arbeitstag am Montag in Nowosibirsk beginnen möchte, muss er schon am Sonntag ausreißen. Die Reisen sind erschöpfend und werden von russischen Berater nicht so gern angenommen.\\
	 Laut dem russischen Rating mit dem Namen "Consulting research" aus 21 größten IT-Consulting-Unternehmen befinden sich 13 Unternehmen in Moskau.%Quelle: http://www.cfin.ru/consulting/rating_uni/it_consulting_2005.shtml 
	 Aus 100 größten russischen IT-Unternehmen befinden sich in Moskau 71 Firmen. 
	 %Quelle:http://www.cnews.ru/reviews/new/rynok_it_itogi_2012/review_table/1d5d1838fd010e16936649555e52b4dd1655219b/
	 Moskau ist nicht nur ein teuerster Hauptstadt der Welt und ein wirtschaftliches Zentrum des Landes, sondern auch ein strategisches Standort für IT geworden.
	 Mit der Stadtwachstum wachsen parallel die Staus.``Nach Angaben des GPS-Navigationsanbieters TomTom ist Mosaku Nummer eins unter den schlimmsten Stau-Städten der Welt.``%Quelle: http://de.ria.ru/society/20130405/265872844.html
	 Da die Berater öfters unterwegs sind, ist es eine große Anstrengung in Moskau Auto zu fahren. Um von A nach B zu kommen wird ganz oft ein Metro benutzt. 
	 Deswegen ist es in Moskau ``erlaubt`` dem Berater halbe Stunde zum Meeting oder zum  Kunde zu spät zu kommen. Allgemein sind die Russen nicht pünktlich, die Termine werden nicht immer eingehalten, E-Mails werden nicht sofort beantwortet und die Versprechungen sind nicht immer realistisch. Deswegen muss man als Berater diese Verzögerungen einplanen.%Quelle: http://www.sekretaria.de/rubrik_GR/geschrakt/Der_Russland-Knigge--A-2715.html
	  \\ 
	  \\
	 	 \textbf{Hierarchie}\\
	 Hierarchien und Entscheidungsfreudigkeit
	 Wenn Verhandlungen anstehen, sollten Sie beachten, dass Entscheidungen oft länger dauern. Mehrere Stellen müssen konsultiert werden bis die Entscheidung fällt. Die Hierarchien sind nicht immer klar erkennbar, dadurch laufen Sie Gefahr, Ihre Zeit in fruchtlosen Gesprächen zu verlieren. Es lohnt sich also noch vor Beginn der Verhandlungen herauszufinden, wer das entscheidende Wort hat.
	 %Quelle: http://www.sekretaria.de/rubrik_GR/geschrakt/Der_Russland-Knigge--A-2715.html
	 ``Flache Hierarchien und ``win-win`` sind nicht die Sache der Russen`` %Quelle:http://www.wissen.de/business-knigge-osteuropa
	 
	\subsection{China}
	
	
	\subsection{USA}
	
	
	\subsection{Deutschland}
	
	
	\subsection{Indien}


\chapter{Bildung, Ausbildung, Forschung}

\section{Einleitung}
Aufgrund des wissensintensiven Charakters des IT-Consultings, ist eine hoch entwickelte Bildungs- und Ausbildungsstruktur Grundvorraussetzung, für das erfolgreiche entstehen eines IT-Consulting Marktes. 
Gibt es beispielsweise nicht genügend Studienabgänger in einem Land, aber einen hohen Bedarf müssen ausländische Fachkräfte zugezogen werden. Die Bildungssituation spielt also auch für den Aspekt Markt eine wesentliche Rolle.

Um Unternehmen angemessen beraten zu können ist auf Seiten der Consultants ein hohes Bildungslevel nötig. Dies ist erforderlich um in angemessener Zeit ein tiefes und breitet IT-Fachwissen aufbauen zu können und die komplexen Zusammenhänge und Interdependenzen erkennen zu können. Außerdem muss auch ein hohes Level an betriebswirtschaftlicher Bildung bei den potenziellen Beratern vorhanden sein um die Unternehmen angepasst auf Ihre Wirtschaftliche Lage beraten zu können.
Die meisten deutschen Unternehmen fordern deswegen von ihren Bewerbern einen Studienabschluss (mind. Bachelor). Es existieren aber auch Länder wie zum Beispiel die Schweiz in der nur ein geringerer Teil der Schulabgänger ein Studium beginnt. In diesen Ländern ist es dann schwieriger eine Verbindung zwischen Studium und IT-Consulting herzustellen, da oftmals auch Personen ohne Studium akzeptiert werden.
Trotzdem stellt ein Studium in vielen Ländern aufgrund der wissensintensiven Tätigkeit des Beratens eine Grundvorraussetzung für die Beratung dar.
Deswegen wird im folgenden die Bildungs- und Ausbildungssituation in den einzelnen Ländern näher betrachtet.

Im Rahmen dieses Projektes beinhaltet jeder der Teilaspekte Aspekt Bildung, Ausbildung, Wissenschaft wiederum verschiedene Teilaspekte oder Kennzahlen die dann jeweils anhand von verschiedenen Ländern verglichen werden. Einige dieser Teilaspekte sind schwierig zu recherchieren, zu bestimmten Aspekten sind auch gar keine Aussagen möglich, dort besteht Forschungsbedarf. 

Teilweise ist es für die einzelnen Teilaspekte schwierig spezifische Daten zum IT-Consulting zu finden, da die Statistiken oft in Ingenieurwissenschaften und Wirtschaftswissenschaften unterteilen. IT-Consulting als interdisziplinär ausgerichtetes Fach ist deswegen oft schwer zuzuordnen. In solch einem Falle wird zum Vergleich eine Durchschnittsgröße aus beiden Studienfächern errechnet.

\section{Bildung}
Der Teilaspekt Bildung beschäftigt sich mit der allgemeinen Bildungssituation im einem Land. Dazu existieren verschiedene bereits Bewertungsverfahren unter anderem auch von der UNESCO. In der Veröffentlichung "World Data on Education" beschreibt die UNESCO die Bildungssituation in den einzelnen Ländern sehr detailliert.
Für die höher entwickelten Länder existieren auch vergleichende Studien der OECD. Die Schwierigkeit besteht jedoch im Vergleich mit Entwicklungsländern, da diese nicht mit zur OECD gehören. 

Einige Teilaspekte die den Bildungsstand eines Landes charakterisieren (diese stammen aus einem Brainstorming des Projektteames):
\begin{itemize} 
\item Anteil der Kinder die eine Schulausbildung machen können
\item Schulpflicht
\item Alphabetisierung
\item Bildungsausgaben
\item Bildungsausgaben Anteil am GDP
\item Anzahl der Schulabgänger ohne Abschluss
\item Qualität der Ausbildung (PISA Studien der OECD, Verschiedene Internationale Rankings)
\item Anteil der Studenten an einem Jahrgang
\item Anzahl der Absolventen Studium und Schule im Verhältnis zur Gesamtbevölkerung
\item Anteil der Hochschulen im Vergleich zur Fläche
\item Anteil der Hochschulen im Vergleich zur Bevölkerung
\item Unterstützung des Staates (vergleichbar mit BAFöG in Deutschland)
\item ... (weitere Punkte können zum Beispiel im Weltbildungsbericht der UNESCO gefunden werden)
\end{itemize}
Eine komplette Ausführung aller dieser Punkte für alle Länder würde den Rahmen dieser Arbeit sprengen, deswegen bezieht sich diese
dieser Teilaspekt in dieser Arbeit nur auf die Studien der UNESCO und der OECD und versucht daraus einen allgemeinen Bildungsstand eines Landes abzuleiten.

Im Folgenden werde einige ausgewählte Teilaspekt und Ihre Relevanz für den Bildungsstand eines Landes erläutert.

\subsection{Anzahl der Hochschulen im Verhältnis zur Fläche (Dichte)}
Die Anzahl der Hochschulen ist für den Bereich der universitären Forschung von großer Bedeutung. Diese prägen maßgeblich die Forschungslandschaft eines Landes mit. Auch für die Ausbildungssituation ist die Anzahl die Hochschulen relevant, denn um so mehr Hochschulen existieren, desto mehr ausgebildete Fachkräfte stehen der Wirtschaft (potenziell) zur Verfügung. 
Trotzdem ist es schwierig nur aufgrund der Anzahl der Hochschulen eine Aussage zu treffen, denn dabei wird die Größe eines Landes außer Acht gelassen. Intuitiv ist klar, das ein größeres Land (bei gleichem Entwicklungstand) im Vergleich zu einen Land das nur halb so groß ist mehr Hochschulen besitzen muss. Deswegen wird in dieser Arbeit die „Dichte“ der Hochschulen anhand des Verhältnissen von  Fläche des Landes zu der Anzahl der Hochschulen berechnet und als Kennzahl gebraucht.
Diese „Dichte“ besitzt jedoch auch einige Schwachstellen. So ist es mit dieser Kennzahl schwierig Länder mit großen dünn besiedelten Gebieten (z.B. Russland) mit kleinen hoch besiedelten Ländern (z.B. Deutschland) zu vergleichen. Allgemein wird einer sehr ungleichmäßigen Bevölkerungsverteilung im Vergleich zur Fläche keine Bedeutung beigemessen.

\subsection{Anzahl der Hochschulen im Verhältnis zur Bevölkerung}
Diese Kennzahl ist sehr ähnlich zum oben genannten Flächenverhältnis. Durch das Verhältnis von Bevölkerung zu Hochschulen ist jedoch die Besiedlungsdichte besser abgebildet als wenn nur die Fläche ins Verhältnis gesetzt wird.
Eine sehr ungleichmäßige Verteilung der Bevölkerung führt jedoch auch bei dieser Kennzahl zu Schwierigkeiten.

\subsection{Bildungsausgaben}
Die Ausgaben die Staat für die Bildung ansetzt sind ebenfalls ein wichtiger Teilaspekt, der die Bildung eines Landes charakterisiert.
Es existieren verschiedene Statistiken zu den Bildungsausgaben. So gibt es z.B. eine Erhebung, die Bildungsausgaben ins Verhältnis zum BIP eines Landes setzt. Diese Vergleichsgröße ist besser geeignet als nur den Geldbetrag für Bildung zu betrachten. Dies erkennt man indem man sich vor Augen führt, das bei gleichem Entwicklungsstand ein größeres Land (mit mehr Bevölkerung) höhere Bildungsausgaben haben muss. Trotzdem das größere Land absolut mehr für Bildung ausgibt, kann es jedoch relativ zur Bevölkerungsanzahl weniger ausgeben als das kleinere Land. Deswegen ist eine weitere Größe nötig zu der die Ausgaben ins Verhältnis gesetzt werden. Dies wird z.B. durch Statistiken zu den Pro-Kopf Ausgaben für Bildung oder durch das Verhältnis von BIP zu Bildungsausgaben erreicht.

\subsection{Unterstützung des Staates} 
Die Unterstützung des Staates für ein Studium spielt für einige Studenten eine wichtige Rolle, da sie es sich sonst nicht leisten könnten ein Studium zu beginnen. Durch staatliche Unterstützung wird es mehr Menschen ermöglicht zu studieren. Dadurch steigen auch die Absolventenzahlen an, somit stehen dem Arbeitsmarkt mehr Arbeitskräfte zur Verfügung.


 \section{Ausbildung}
Dieser Teilaspekt bezieht sich im Gegensatz zum eher allgemeinen Teilaspekt Bildung stärker auf das IT-Consulting. In diesem Abschnitt werden die spezifischen IT-Consulting Ausbildungsmöglichkeiten eines Landes näher beschrieben.
Da aber oftmals keine spezifischen Daten zu IT-Consulting Studiengänge oder Ausbildungen vorhanden sind (da es diese spezielle Ausrichtung oft nicht gibt) muss teilweise mit Fächergruppen gearbeitet werden.
Dadurch ergibt sich das Problem der Einordnung von IT-Consulting in die Fächergruppen Ingenieurwissenschaften und Wirtschaftswissenschaften/Rechtswissenschaften. Diese Zuordnung ist nicht möglich (interdisziplinäre Wissenschaft) deswegen wird mit einem Durchschnitt gearbeitet.

Teilaspekte die relevant sind:
\begin{itemize} 
\item Welche IT-Consulting Studiengänge existieren?
\item Welche anderen Studiengänge sind für IT-Consulting relevant?
\item Wie viele Absolventen gibt es in den relevanten Studiengängen?
\item Welche anderen Ausbildungsformen existieren für IT-Consulting?
\end{itemize}

Einige dieser Aspekte und ihre Relevanz für das IT-Consulting werden im folgenden erläutert.

\subsection{Relevante Studiengänge}
Wie bereits erwähnt erscheint es schwierig, IT Consulting zu einem Studienbereich oder einem Studiengang zuzuordnen. 
In Deutschland existieren vereinzelt aber auch Bachelor und Masterstudiengänge die Consulting im Namen tragen, so z.B. der Master Studiengang der Universität Hamburg. Ein genauerer Überblick über die in Deutschland, Österreich und der Schweiz verfügbaren spezifischen Bachelor Studiengänge wird in \cite{NissenKlaukDeelmannMohe201209} gegeben. Trotzdem stellen diese Absolventen nur einen kleinen Teil der benötigten Fachkräfte dar. 

Es erscheint logisch, das die Studienrichtung Wirtschaftsinformatik durch die interdisziplinäre Ausrichtung gut zum IT-Consulting passt. Auch die Studiengänge Betriebswirtschaftslehre sowie Informatik erscheinen für eine IT Consulting Karriere geeignet. Welche weiteren Studiengänge relevant sind lässt sich nur schwer ermitteln, das es keine Statistiken über die Herkunft der Berufseinsteiger im IT-Consulting gibt.

\subsection{Absolventen in relevanten Studiengängen}
Dieser Teilaspekt setzt natürlich eine Auswahl an relevanten Studiengängen oder Ausbildungen voraus. Die Anzahl der Absolventen wäre dann die potentielle Menge die dem Arbeitsmarkt zur Verfügung stehen würde.Werden besonders viele Ausländische Fachkräfte für die Beratung eingestellt, kann man auf einen Mangel an eigenen ausgebildeten Fachkräften schließen und somit auf einen höheren Bedarf.

\section{Forschung}
Der Bereich Forschung beschreibt die für das IT-Consulting relevanten Forschungen. Wiederrum ist es schwierig eine genaue Zuordnung von Forschungsaktivitäten in verschiedenen Fakultäten zur Beratungsbranche zu treffen. Zu diesem Teilaspekt besteht der größte Forschungsbedarf, da bisher nur wenig relevante Forschung betrieben wird.

Einige Teilaspekte zur Forschung sind beispielsweise:
\begin{itemize}
\item Verhältnis Forschung zur Praxis im IT-Consulting
\item Anzahl der wissenschaftlichen Veröffentlichungen die das IT Consulting betreffen
\item Forschungsgelder und Subventionen
\end{itemize}

Einige dieser Aspekte und ihre Relevanz für das IT-Consulting werden im folgenden erläutert.

\subsection{Anzahl der wissenschaftlichen Veröffentlichungen}
Dieser Teilaspekt betrifft die Anzahl der wissenschaftlichen Veröffentlichungen die sich mit IT-Consulting beschäftigen. 
Für Deutschland existiert eine vom statistischen Bundesamt veröffentliche Erhebung zur Anzahl und Fächerverteilung der Promovierenden.\cite{destatis}
Auch bei dieser Statistik ergibt sich die Schwierigkeit der Zuordnung eines Studienbereiches (Wirtschaftswissenschaften oder Ingenieurwissenschaftlich?) zum IT-Consulting. Ähnliche Statistiken über Bachelor und Master Arbeiten oder für andere Veröffentlichungen existieren für Deutschland nicht. 

\subsection{Verhältnis Praxis zu universitärer Forschung}
In einigen Veröffentlichungen wird IT-Consulting als rein praktische Tätigkeit betrachtet und somit der Sinn einer wissenschaftlichen Erforschung angezweifelt. Dem entgegen steht jedoch die Forschungsrichtung „Consulting Research“ (Volker Nissen). Consulting Research besitzt laut Nissen zwei Ziele: „Erstens, die wissenschaftliche Durchdringung des Themas Unternehmensberatung, wobei der von einzelnen Beratungsprojekten abstrahierende wissenschaftliche Erkenntnisgewinn im Mittelpunkt steht. Zweitens, die Übertragung wissenschaftlicher Theorien, Erkenntnisse und Methoden auf die unternehmerische Praxis mit dem Ziel, Aufgabenstellungen und Probleme im Umfeld von Beratungsprozessen und Beratungsunternehmen besser als bisher zu lösen. “
(Ausführliche Diskussion des Verhältnisses von Praxis zu Theorie auf S.23 ff)

Nissen erkennt auch ein „ Defizit an wissenschaftlicher Auseinandersetzung [...] mit Themen der IT-orientierten Beratung“. Dies erschwert es eine Aussage zum Forschungsstand im Consulting zu machen.

Diese Arbeit folgt der Auffassung von Nissen, dass eine wissenschaftliche Durchdringung des Consultings nutzbringend ist.

\subsection{Staatliche Forschungsgelder/Subventionen/Zuschüsse}
Dieser Teilaspekt beschäftigt sich mit der Höhe der Forschungsgelder die dem IT-Consulting zugeordnet werden können.Durch höhere Forschungsgelder können mehr Forscher eingestellt werden und mehr Zeit investiert werden, dies kann zu einer höheren Qualität der Ergebnisse führen. Forschungsgelder sind demnach relevant sowohl für die Menge der Forschungsarbeiten als auch für deren Qualität.
Für Deutschland existieren z.B. Statistiken die staatliche Forschungsgelder nach Fakultäten auflistet (sowohl für Universitäten als auch für Fachhochschulen). Erneut ist die Zuordnung des IT-Consultings zu einem der Fachbereiche Ingenieurwissenschaften oder Wirtschafts-/Rechtswissenschaften nur schwer möglich.


\section{Deutschland}
\subsection{Bildung}
\subsection{Ausbildung}
\subsection{Forschung}

\section{USA}
\subsection{Bildung}
\subsection{Ausbildung}
\subsection{Forschung}

\section{Indien}
\subsection{Bildung}
\subsection{Ausbildung}
\subsection{Forschung}







