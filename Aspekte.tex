%!TEX encoding = UTF-8 Unicode
\chapter{Aspekte}
\chapter{Markt}
\section{Einleitung}
Im folgenden Kapitel soll ein grober Überblick und eine Einleitung zum IT-Consulting Markt gegeben werden. 
Ziel einer solchen Recherche sowohl Informationen zu sammeln, die strategische Möglichkeiten für Expansion, Kooperation oder Offshoring (Outsourcing ins Ausland)  aufzeigen, 
als auch die Wettbewerbssituation besser einzuschätzen, um daraus Handlungsalternativen abzuleiten. 
In zahlreichen Studien tauchen in diesem Zusammenhang häufig sehr weitgefasste Begriffe auf wie „technology industry“ oder "IT-service-market" auf. 
Die Grenzen der Bandbreite der darin enthaltenen Beratungsleistungen sind darum besonders unscharf. Eine Differenzierung ist aufgrund der Komplexität der Zusammensetzung des Dienstleistungsspektrums daher nur im groben Umfang möglich.
Deswegen wurde darauf geachtet, dass die erfassten Unternehmen einen Mindestanteil  für ausschließlich beratende Tätigkeiten von mindestens 40\%  aufweisen, um in eine IT-Consulting äquivalente Kategorie zu fallen.
Es gibt zahlreiche Schlüsselfaktoren, die für die Einschätzung des Marktes in den jeweiligen Ländern bzw. Kontinenten aufschlussreich sind. 
Allerdings ist eine Erhebung sehr zeitaufwändig oder/und teuer. Es gibt zahlreiche Marktstudien von großen Marktforschungsunternehmen, wie z.B. Gartner, die man ab ca. 1000 €  erwerben kann, die aber nur Teilaspekte abdecken. 
Es gibt kaum kostenfreie umfassendere Studien zu dem gesamten Thema, dafür jedoch zahlreiche kostenpflichtige Studien zu hohen Preisen. Es lässt sich daher vermuten, dass es zumindest aufgrund des niedrigen Angebots an kostenlosen Studien, eine moderate Nachfrage nach kostenfreien Marktinformationen im Bereich des IT-Consulting gibt. Ob die kostenpflichtigen Studien den Informationsbedarf abdecken würden, gilt es daher zu prüfen und zu erwägen, ob eine Investition darin lohnt.

 Ziel des Kapitels ist es, sich dem  Thema zuerst theoretisch anzunähern und die Schlüsselfaktoren zu überlegen und deren konkrete Größen zu ermitteln. Anschließend erfolgt eine Untersuchung einzelner Teilaspekte.

\section{Teilaspekte}
Nachfolgend soll zuerst einmal begründet werden, welche Teilaspekte für eine Marktrecherche im IT-Consulting für eine Studie als besonders relevant erachtet werden.
Ein Großteil der Arbeit liegt deswegen vor allem in der Ermittlung des Informationsbedarfes um zu entscheiden, welche Kenngrößen als Faktoren überhaupt relevant sind.
Sichere Prognosen können daraus nicht in jedem Fall resultieren, da die gewählten Teilaspekte und deren Faktoren auf vernünftige Schätzungen beruhen und die Marktaspekte dynamische Faktoren eines komplexen und chaotischen System sind. Es lassen sich auf dieser Basis  jedoch vernünftige Schlussfolgerungen ziehen, die es wiederum zu verifizieren gilt. Statistische Verfahren bieten hier eine Möglichkeit, um zumindest eine hohe Wahrscheinlichkeit zu erreichen. 
Wie derart vage Informationen in eine Entscheidung einbezogen werden, bleibt außen vor. Es gibt jedoch zahlreiche Verfahren in der Entscheidungstheorie, welche unsichere Umstände in Managemententscheidungen einbeziehen. Letztendlich kommt es auf den Anwendungszweck der Information an, ob diese für eine Entscheidung relevant ist.
 
 \subsection{IT-Consulting Gesamtumsatz und Marktwachstum je Land}
 Diese Kennzahlen liefern wichtige Hinweise, wie sich die Branche weiter entwickeln wird und wie sehr sie schon entwickelt ist. 
 Diese Daten dienen dazu um Rückschlüsse auf die Nachfrage von IT-Consulting-Leistungen zu ziehen. 
 Dazu werden natürlich noch eine Reihe weiterer Daten benötigt, wie Gesamtmarktvolumen und Gesamtwirtschaftswachstum, um die Kennzahlen ins Verhältnis zu setzen.
  Außerdem muss darauf geachtet werden, welche Umstände die Kennzahlen verfälschen wie z.B. die Ländergröße, wodurch eher ein höherer Umsatz pro Land entsteht. 
  Es werden deswegen weitere Kennzahlen wie z.B. Fläche des Landes oder Einwohnerzahlen benötigt, um diese Faktoren ordnungsgemäß bewerten zu können.
 Der IT-Consulting-Umsatz und das Wachstum je Land ist im Verhältnis zu anderen Kenngrößen durchaus im vertretbaren Umfang anhand von öffentlichen Studien erfassbar.  
 Daher sollen diese Größen einzeln nach Ländern aufgelistet und erläutert werden. 
  Dabei ist zu berücksichtigen, dass auf die daraus resultierenden Anteile und Kennzahlen, dem Anspruch auf Korrektheit nicht gerecht werden kann. 
  Hierfür müssen diese Kennziffern noch mehrfach überprüft und Einflüsse die zur Verwässerung der Faktoren führen, herausgerechnet werden, um eine möglichst genaue Einschätzung zum Marktanteil und dem Wachstum zu treffen. Da dies jedoch den Rahmen der Arbeit sprengen würde, wird zunächst eine vereinfachte Betrachtung vorgenommen. Diese kann bei Bedarf als Ausgangsbasis  verwendet werden, um eine korrigierte Aussage zu treffen. 
  Für einen generellen Überblick und eine grobe Quantifizierung sind diese Daten essenziell. So können auf Basis dieser weitere Vermutungen angestellt und entsprechende Recherchen unternommen werden.
 Es wurde sich dazu entschieden diesen Teilaspekt für die Recherche auszuwählen. (siehe  \ref{subsubsec:Gesamtumsatz} \nameref{subsubsec:Gesamtumsatz} auf Seite \pageref{subsubsec:Gesamtumsatz})
 
\subsection{Gewinn- und Umsatzzahlen von Großunternehmen weltweit/Länder spezifisch}
Diese Daten sind gut zugänglich und liefern eine grobe Richtzahl über Umsatz und Erfolg in der Branche in dem jeweiligen Land. Hauptziel ist es eine Nachfrage aus den Kennziffern abzuleiten. So lassen sich anhand des Vergleiches der Umsatzzahlen und Gewinne der gleichen Unternehmen in verschiedenen Ländern Rückschlüsse auf mögliche Ursachen ziehen.
 Es muss hierbei jedoch berücksichtigt werden, ob die Großunternehmen nur Großprojekte übernehmen oder auch kleine bis mittlere Projekte. Sollte der hohe Umsatz maßgeblich durch Großprojekte generiert werden, muss dies nicht gleichermaßen kleine und mittlere Unternehmen gelten.
Des Weiteren dienen diese Daten dazu, die erfolgreichsten Unternehmen zu ermitteln, um diese anschließend zu beobachten und deren Wettbewerbsvorteile zu erkennen. Dieses Wissen liefert Hinweise auf Indikatoren, die zu dem Erfolg eines Unternehmens in dem jeweiligen Land beigetragen haben.
 So kann ein Unternehmen, welches z.B. auf SAP spezialisiert ist, zwar in Brasilien erfolgreich sein, jedoch in China trotz guter Marktlage nicht. Ursachen liegen in diesem Fall nicht im Markt, sondern z.B. in der strategischen Ausrichtung der IT in chinesischen Unternehmen (z.B. Kostendruck, kurzsichtige Denkweise) oder weil es sehr starke Konkurrenten gibt.
Signifikante Unterschieden bieten Potential um Vermutungen aufzustellen und diese weiter zu untersuchen oder um allgemeine Unterschiede im Erfolgspotential zwischen Ländern abzuleiten. 

 \subsection{Firmengrößen im IT-Consulting}
Ein wichtiger Einflussfaktor um die Konkurrenzsituation festzustellen, ist die Beschaffenheit des Marktes nach Unternehmensgrößen.
 Es stellt sich die Frage, ob eher Großkonzerne, mittelständische und kleine  Unternehmen oder gar interne Mitarbeiter für die strategische und ganzheitliche Gestaltung der IT beauftragt werden.
Es kann zum einen die durchschnittliche Firmengröße eines IT-Consulting Unternehmens herangezogen werden  und zum anderen die prozentualen Anteile der einzelnen Sektoren.  Es gibt Länder, die eher einen breiten Mittelstand haben oder eher von Großkonzernen dominiert werden. 
Ziel einer solchen Analyse ist es, Häufungen oder Bedarfe in einzelnen Sektoren festzustellen. 
Beispielsweise ein mittelständisches Unternehmen könnte, wenn es nur einen geringen Anteil kleiner und mittelständischer Unternehmen gibt, von einem erhöhten Bedarf in der IT-Beratung mit kleiner bis mittlerer Projektgröße profitieren, in so fern die Großunternehmen nur an großen Projekten interessiert sind.

Die Möglichkeiten zur Ermittlung der Marktbeschaffenheit bestehen aus  folgenden Schritten:
\begin{itemize}
\item Statistische Stichproben / Befragungen zur Ermittlung der Anteile
\item Analyse Dienstleistungsangebote der IT-Consulting-Unternehmen in dem jeweiligen Sektor 
\item Berechnung der Marktanteile der größten Unternehmen und Bewertung des Restwertes zum Vergleich mit der Stichprobe
\item Beschaffung von Studien oder Beauftragung eines Marktforschungsunternehmen, falls die Stichproben nicht ausreichen
\end{itemize}
Die resultierenden Informationen zur Beschaffenheit geben wertvolle Hinweise über das Angebot in der Branche des IT-Consulting. 
So können von diesem Wissensstandpunkt aus die konkreten Angebote analysiert werden und ggf. Chancen abgeleitet werden. 
Beispielsweise können große Unternehmen aufgrund ihrer Kosteneffizienz möglicherweise bessere Qualität anbieten als ein breiter Mittelstand, gleichzeitig die Nachfrage nach mittelständischen Unternehmen aufgrund günstigerer Preise höher sein. 
Eine tiefgehende Analyse ist jedoch sehr komplex und würde Stoff für eine eigenständige Studie liefern und soll deswegen kein weiterer Gegenstand sein.

 \subsection{Politik / Rechtslage}
Die politische und rechtliche Lage ist ein wichtiger Faktor für die Wahl eines Standortes oder die Zusammenarbeit mit einem Land. Gleichzeitig ist es ein Teilaspekt, der die Entwicklung der Branche beeinflusst.
Besonderen Einfluss haben die folgenden Faktoren auf die IT-Consulting Branche:
\begin{itemize} 
\item {generelle staatliche Subventionen / Investitionen in die IT}

 Staatliche Förderprogramme und Subventionen für IT-Entwicklungen stellen eine wichtige Finanzierungsmethode für Entwicklungsprojekte oder Unternehmen, deren Existenz bedroht ist, dar. Diese Subventionen bestehen beispielsweise aus zinsgünstigen Darlehen oder Fördermitteln, die nicht zurückgezahlt werden müssen. Die unterschiedlichen Arten von Fördermitteln müssen daher klassifiziert und entsprechend eingeordnet werden
 Solche finanziellen Unterstützungen treiben natürlich die Forschung und den allgemeinen Wissenstransfer voran. Insbesondere das IT-Consulting ist eine reine Wissensbranche. Wenn die IT-Entwicklung von Unternehmen profitiert, profitiert auch das IT-Consulting von den neuen Erkenntnissen. 
 Staatliche Finanzierungsformen können als Anreiz für Unternehmensgründungen dienen und damit die Innovativität der IT-Landschaft fördern. Neue Technologien und Erkenntnisse führen zu neuen Handlungsalternativen. Das Wachstum in der IT-Branche erhöht wiederum die Komplexität von Entscheidungsprozessen und Nachfrage nach Beratungsleistungen.
Ein Wissen über Staatssubventionen kann daher hilfreich sein, um das Entwicklungspotential und das Maß an finanzieller Stabilität für Unternehmens-Neugründungen, welche durch Förderprogramme erhöht werden kann, einzuschätzen.
 \\
\item  {Handelsrecht / Arbeitsrecht}

 Das Vertrags-und Arbeitsrecht hat Einfluss auf das Outsourcing oder die Kooperationen mit einem ausländischen Unternehmen. 
 So sind in jedem Land bestimmte Handels und Arbeitsgesetze zu beachten, die eine reibungslose Unternehmens-Kooperation oder ein rechtmäßiges Arbeitsverhältnis gewährleisten. 
Für das Unternehmen steht vor allem die Frage nach der Haftung und Behandlung von Mängeln im Vordergrund. 
Eine Einordnung nach Risiken und Chancen ist hier folglich notwendig, um das Land zu beurteilen. 
 \\
\item {Steuerrecht}

 Das Steuerrecht ist vor allem für die Standortwahl ausschlaggebend. 
 So sind möglicherweise bestimmte Besteuerungsvorschriften mit in die strategische Standortauswahl einzubeziehen. So gilt es abzuwägen, ob die Gewinnerwartungen nach Steuern höher sind, als in einem anderen Land. Zahlreiche Unternehmen suchen sich Ihren Hauptsitz daher nach den für Sie günstigen Steuervorteilen aus.
 Allerdings gilt es verschiedene Punkte zu analysieren wie:
 - Höhe der Mehrwertsteuer
 - Legalität bei Dienstleistungsvertrieb und Aufenthalt in einem anderen Land
 - Höhe der Einkommensteuern
 - Höhe der Gewerbesteuer/Grundsteuer
 - Spezielle Sonderregelungen
  \\
\item {Datenschutz/Urheberrecht}

  Der Datenschutz ist ausschlaggebend für den Austausch von sensiblen Unternehmensdaten. 
  Die Gefahr von Produktpiraterie oder der Weitergabe von sensiblen Daten kann ein Risiko für Kooperation mit Partnern, Expansion oder dem Outsourcing sein. Ursachen liegen hier in zu schwachen oder gar nicht vorhandenen Gesetzen für das Urheberrecht oder dem Datenschutz.
  
  \end{itemize}

\section{Analyse ausgewählter Teilaspekte}
Aufgrund des hohen Rechercheumfangs im Bereich Markt wurde sich auf den Teilaspekt "IT-Consulting Gesamtumsatz und Marktwachstum je Land" beschränkt. Dafür wurde besonders auf Vollständigkeit der Daten für einen Vergleich geachtet. 
\subsection*{IT-Consulting Gesamtumsatz und Marktwachstum je Land}
\label{subsubsec:Gesamtumsatz}

\begin{itemize} 
\item {Global}

Insgesamt wurden 2010 laut Gartner 574,94 Milliarden Euro weltweit in der IT-Consulting-Branche umgesetzt. \cite{itConsultingGlobal} Das durchschnittliche globale jährliche Wachstum beträgt 2,6\% zwischen 2007 und 2011.\cite{globalGartner}

\begin{figure}
  \centering
  \includegraphics[width=0.8\textwidth]{images/global_revenue_share.jpg} 
  \caption{Anteile am IT-Consulting Weltmarkt} \label{fig:weltmarkt} 
\end{figure}


\item {Deutschland}

Das Wachstum des deutschen IT-Consulting ist mit 8,4\% sehr hoch im Verhältnis des Wirtschaftswachstums von 3\% in 2011.\cite{statGer2} Dabei haben die Top 25 IT-Consulting- Unternehmen sogar noch ein größeres Wachstum mit Spitzen bis zu 10\%. \cite[6]{topITB} Dies sind durchaus überdurchschnittliche Wachstumsraten, welche international mit aufsteigenden Ökonomien wie Indien, Russland oder China sehr gut mithalten können.  (siehe  \ref{table:umsaetze} \nameref{table:umsaetze} auf Seite \pageref{table:umsaetze}) Der Gesamtumsatz des IT-Consulting beträgt 29,4 Milliarden Euro.  Das sind 1,12 Prozent des gesamten bereinigten Bruttoeinlandproduktes. \cite{statGer} Um die Dichte des IT-Consulting-Marktes einschätzen zu können macht es Sinn den Gesamtumsatz auf die Ländergröße zu beziehen. 
Deutschland erzielt 8,24 Milliarden Euro pro 100000 km² ab. Dies ist im Vergleich mit den anderen Ländern eine enorm hohe Dichte und bietet dadurch eher Wegersparnisse und Kommunikationsvorteile, die sich strategisch günstig auswirken können. 

\item {USA}

Die USA zweifellos einen gigantischen Marktanteil an IT-Services. Mit 244 Milliarden Euro, was 43\% des weltweiten IT-Service Markt einnimmt, ist es mit Abstand das umsatzstärkste Land der Welt im Bereich IT-Services. \cite{ibisUSA} Das Wachstum stagniert allerdings mit 2,2\%. Im Verhältnis zum Wachstum des bereinigten Bruttoninlandsproduktes von 1,8\% wächst es nur wenig mehr. \cite{statUSA} Die Dichte des IT-Consulting-Marktes beträgt 2,48 Milliarden Euro pro 100000 km². Im Verhältnis zu anderen Ländern wie z.B. Indien oder Russland ist dies eine sehr hohe Dichte, nur Deutschland schneidet noch deutlich besser ab.

Ein Großteil der Umsätze in den USA entsteht unter anderem dadurch, dass Wertschöpfung durch IT-Services, die im Ausland durch Offshoring entstehen, hinzugerechnet werden. Diese Offshoring-Länder sind daher einer der Schlüsselfaktoren für die hohen Umsätze in den USA. 

\item {China}

China hat mit 72,6 Milliarden Euro den zweitgrößten Marktanteil der Welt. Es hat zwar noch weniger als ein Drittel gegenüber den USA, jedoch verzeichnet es mit 6,8\% in 2011 überdurchschnittliche Wachstumsraten im Bereich IT-Services und hat damit das dreifache Wachstum des Konkurrenten USA.  Im Verhältnis zum Wachstum des Bruttoinlandsproduktes mit 9,2\% in 2011 ist das Wachstum jedoch verhältnismäßig wenig. Dieses Verhältnis deutet stark daraufhin, dass die strategische Ausrichtung der IT und deren Prozessen noch nicht gleichermaßen Beachtung geschenkt wird, wie anderen Dienstleistungen und Produkten. Mögliche Ursachen könnten Fachkräftemängel oder Kostendruck sein. Es gilt daher weiter zu untersuchen, welche Ursachen das verhältnismäßig schwächere Wachstum hat. 
Die Wachstumsraten im Bereich IT-Services sind in alle Ländern über dem des BIP. Warum dies in China so weit nach unten abweicht ist, gilt es daher weiterhin zu untersuchen und dafür Ursachen zu finden.
Die Dichte ist im Verhältnis zu Deutschland oder den USA auch deutlich schwächer. Hier schneidet China mit einer IT-Service-Dichte von 0,74 Milliarden pro 100000 km² ab. Da China sehr weitläufig und nicht vollständig industrialisiert ist, ist diese Größe wenig aussagekräftig und es besteht weiterer Untersuchungsbedarf, ob diese Größe wirklich auf einen schwächer entwickelten IT-Consulting-Markt hindeutet oder anderen demografischen Umständen geschuldet ist. \cite{ibisChina}

\item {Russland}

Der russische Markt im Bereich IT-Services ist mit 14,3 Milliarden recht klein wenn man es auf die Größe des Landes bezieht. Es gibt vor allem viel System- und hardwarenahe Entwicklung. Dienstleistungen im IT-Sektor wachsen jedoch in zunehmenden Maße und konnten 2011 15,3\% Branchen-Wachstum erreichen. Dies ist sowohl im Vergleich zum internationalen Markt als auch im Verhältnis zum Wachstum des Bruttoinlandsproduktes mit 4,3 \% in 2011 weit überdurchschnittlich. \cite{statRus2} Aufgrund der relativ kostengünstigen Entwicklungskosten für Software, insbesondere hardwarenahe Entwicklung und Systemengineering wird Russland vor allem in Europa zunehmend als „Offshoring Land“ attraktiv. (Wirtschaftsinformatik und Management 12/2013, Offshoring Land Russland)
Die Dichte des IT-Consulting fällt mit 84 Millionen Euro pro 100 000 km² sehr klein aus. Dabei ist zu berücksichtigen, dass ein Großteil von Russland gar nicht oder nur schwach bewirtschaftet ist. \cite{statRus}


\item {Afrika}

Afrika hat einen sehr niedrigen Anteil am Markt mit 1,4 Milliarden. \cite{statAfr} Statistiken zum Wachstum konnten leider nicht gefunden werden. Die großen Technologie-Beratungs-Konzerne haben sich in verschiedenen Teilen Afrika als Beratungen etabliert, die auch den IT-Consulting Markt abdecken. Der Trend geht jedoch laut Experten immer mehr dahin, dass neue inländische IT-Service-Provider auf dem Markt konkurrieren und die US-Konzerne ablösen. Afrika hat 2011 mit 5\% ein durchaus gutes Wirtschaftswachstum zu verzeichnen, kann aber nicht mit anderen aufstrebenden Ökonomien mithalten, insbesondere nicht im IT-Umfeld. \cite{statAfr2}
Ursachen liegen hier nicht zuletzt in der Stromversorgung. Denn 80\% der Dörfer in Afrika sind immer noch ohne Stromversorgung, weil Energieanbietern die Vernetzung der Dörfer zu teuer ist.  \cite{dieZeit}
Die Situation des Marktes im gesamten Kontinent ist jedoch sehr vielschichtig. Eine Analyse des Marktes in sehr komplex, da in Afrika sehr viele demografische und infrastrukturelle Umbrüche stattfinden. Daher ist hier eine Betrachtung der Umsätze wenig aussagekräftig für eine klare Einschätzung. 
Um eine besser Bewertung zu ermöglichen, ist eine Unterteilung von Afrika notwendig. Des weiteren sind die Kennzahlen, aufgrund der hohen Marktdynamik, aus nur einem Jahr nicht ausreichend aussagekräftig. Deswegen ist es sinnvoll die Umsätze und Wachstumsraten, noch ein einem größeren Zeitraum zu betrachten.
Es besteht daher hier weiterer Forschungsbedarf, um die Kennzahlen in einem differenzierteren Kontext zu stellen.
Die Werte eignen sich daher mehr für den Vergleich der IST-Situation mit anderen Ländern.


\item {Brasilien}

Brasilien hat im Jahre 2011 mit einem Branchenumsatz von 69,6 Milliarden, den drittgrößten Anteil am globalen IT-Consulting Markt erzielt. Das Wachstum über die Jahre von 2008-2011 beträgt 61\%. \cite{statBras2} Trotz seiner flächenmäßigen Größe weist es mit 0,82 Milliarden pro 100 000 km eine verhältnismäßig hohe Dichte auf. Das Wachstum in 2011 in der IT-Services-Branche beträgt hier 4,9\% und ist damit weitaus höher als das Gesamtwirtschaftswachstum mit 2,7\%. Dabei beträgt der Anteil am Bruttoinlandsprodukt allein 4,5\%. Dies zeigt welchen großen Stellenwert der Markt für IT-Services in Brasilien einnimmt. Experten prognostizieren weiterhin einen rasanten Anstieg des Wachstums, insbesondere im Bereich BPO, welche laut Schätzungen 85\% am Gesamtmarktanteil einnimmt.\cite{statBras}

\item {Zusammenfassung}

Alle Werte beziehen sich auf die Jahre von 2011 bis 2012. Die entsprechenden Quellen sind in den Länderabschnitten zu finden.


\begin{table}

\caption{Übersicht Umsatz und Umsatzwachstum von IT-Services im Verhältnis zum BIP  (2010)}
\begin{tabular}{|p{2.6cm}|p{1.5cm}|p{2cm}|p{1.5cm}|p{1.5cm}|p{1.7cm}|}
 \hline
  \textbf{Land} & \textbf{Umsatz in Mrd. €} & \textbf{IT-Consulting Wachstum} & \textbf{BIP Wachstum} & \textbf{Welt-Markt-Anteil} & \textbf{Umsatz-Dichte in Mrd. € pro 100 000 km²} \\
  \hline
    
    1. USA  & 244,33  &2,2\%  & 1,8\% & 43\% & 2,48  \\
    2. China & 72,50 & 6,8\%  & 9,2\% & 13\% & 0,74 \\
    3. Brasilien & 69,60 & 4,9\%  & 2,7\% & 12\% & 0,817 \\
    4. Deutschland & 29,4 & 8,5\%  & 3\% & 5\% & 8,24 \\
    5. Indien & 25,45  & 11.2\%  & 7.9\% & 4\% & 0,77  \\
    6. Russland & 14,3  & 15.4\%  & 4,3\% & 2\% & 0,084  \\
    7. Afrika & 1.4  & n/a  & 5\% & 0.5\% & 0,0046 \\
 \hline
\end{tabular}
\label{table:umsaetze} 
\end{table}

\end{itemize}




%\section{Arbeitskultur}
	\section{Einleitung}
Bevor die Wichtigkeit der Arbeitskultur für die Beratungsdienstleistung erläutert wird, wird an dieser Stelle auf die Definition der Arbeitskultur eingegangen. Reinhard Kößler definiert den Begriff der Arbeitskultur im ``Lexikon zur Sozialogie`` von Wernern Fuchs-Heinritz als Verschiedenheit von Visionen des Arbeitsverhaltens in Form von Lebensformen, Einstellungen und Reaktionen auf die Anforderungen der Arbeit in industriell-kapitalistischen Gesellschaft. \cite{Fuchs-HeinritzLautmannRammstedtWienold1994}.
Arbeitskultur ist in der ersten Linie eine Teilmenge der Kultur (Sitten, Bräuche, Mentalität usw.) einer Nation. Gemäß Carsten Weigelt bedeutet die Arbeitskultur für die jungere Generation - sogenannten ``Digital Natives`` viel mehr als Geld und Karriere. Unter Arbeitskultur wird von ihnen das Wohl im Privatleben und am Arbeitsplatz verstanden.
Viel wichtiger ist an dieser Stelle, anstatt dem Begriff von Fuchs-Heinritz, die letzte Definition der Arbeitskultur von ``Digital Natives``zu betrachten. Denn für den Beratungsprozess ist es viel sinnvoller die gewünschte und die gute Arbeitskultur zu betrachten. Um festzustellen was eine gute Arbeitskultur für die IT-Beratung im internationalem Kontext ist, wurden einige Teilaspekte der Arbeitskultur definiert und abschließend werden diese Teilaspekte anhand von ausgewählten Ländern verglichen. Was beeinflusst die gute Arbeitskultur? welche Faktoren führen zum Wohl im Privat-und Berufsleben? Ist es überhaupt wichtig die fremde Arbeitskultur zu analysieren, wenn man international agiert? In wieweit sind die Beratungsdienstleistung von der Arbeitskultur abhängig? Diese und weitere Fragen rund um IT-Beratung, sowie Arbeitskultur im internationalem Kontext werden in den nächsten Abschnitten geklärt. Es werden auch einige Probleme rund um die Arbeitskultur, IT-Beratung und den internationalen Aspekt der Arbeitskultur erläutert.\\
Die Arbeitskultur gehört zum Beratungsprozess und spielt dabei nicht die unwesentlichste Rolle. Welche Arbeitskultur gehört zum Beruf des IT-Beraters? Ein IT-Berater ist immer in der Bewegung und sein Arbeitsplatz ist nicht nur im Büro, sondern auch im Zug, im Restaurant oder im Auto. Es ist sehr ersichtlich, dass die Arbeitskultur des Beraters mit der Arbeitskultur von Kunden des Beraters unzertrennlich ist. IT-Consultans lernen innerhalb der wenigsten Zeit sehr viele Firmen und deren Mitarbeiter kennen. An dieser Stelle stoßen die IT-Berater auf unterschiedlichste Arbeitskulturen. Berater arbeiten oft durch Kommunikation mit Menschen aus unterschiedlichen Unternehmensebenen (Mitarbeiter, Manager, Geschäftsführer usw.), verschiedenen Branchen (Finanzdienstleistung, Fahrzeugbau, Großhandel, Chemieindustrie usw.) oder in unterschiedlichen Länder mit jeweils einzigartigen Kulturen und damit auch einzigartigen Arbeitskulturen.\\
Im Vergleich zu einem Mechatroniker, der nur eine Arbeitskultur ``kennt``, werden die IT-Berater mit unterschiedlichsten Arbeitskulturen konfrontiert, häufig auch international.\\
Um die Bedeutung der Arbeitskultur im internationalem Kontext für den Beratungsprozess näher zu erläutern, werden an dieser Stelle 2 Beispielfälle erklärt.\\
	 \\
	 a) Der 1. Fall ist ein IT-Consulting-Unternehmen mit eingestellten Beratern, die aus unterschiedlichen Ländern kommen, unterschiedliche Sprachen sprechen und sich kulturell enorm unterscheiden. Wichtig für die Arbeitskultur an dieser Stelle ist es ein kulturelles Gleichgewicht herzustellen und dauerhaft zu behalten. Diese Berater arbeiten zielgerichtet und ständig im Team. Im 1. Fall ist es vor allem  interessant, inwieweit sich kulturellen Unterschiede auf das gemeinsame Ziel des Beratungsprozesses bei der Softwareeinführung auswirken können. Auch interessant ist hier, wie die IT-Berater aus unterschiedlichen Länder mit Kunden aus Deutschland umgehen und ob die kulturelle Unterschiede einen Einfluss auf Kundenbeziehungen haben. \\
	 \\
	 b) Der 2. Fall bezieht sich auf ein deutsches Unternehmen, das international agiert und Kunden aus unterschiedlichen Länder betreut. In diesem Fall müssen sich deutsche Mitarbeiter auf unterschiedliche Arbeitskulturen anpassen. Denn ein Meeting während des Mittagsessen in Japan ist widersinnig und wirkt unseriös. In Japan hat das Essen einen unverletzlichen Status, in USA dagegen ist es nicht ungewöhnlich, dass beim Essen wichtige Entscheidungen kollaborativ getroffen werden.\\
	Wegen der zeitlichen sowie thematischen Begrenzung wird in dieser Arbeit der Fokus nicht auf die Differenzierung dieser zwei Fälle oder auf kulturelle Unterschiede der Berater gelegt, sondern nur auf die unterschiedliche Arbeitskulturaspekte, die für den Beratungsprozess ausschlaggebend sind. Teilaspekte der Arbeitskultur, die für das IT-Consulting als relevant erachtet werden, werden in folgenden Kapiteln vorgestellt und verglichen. In diesem Sinne werden diese zwei Fälle nicht unterschiedlich und nur im Hintergrund behandelt.
	Diese sind aber wichtig, um zu zeigen, warum sich die IT-Berater-Arbeitskultur nicht nur auf die deutsche Arbeitskultur bezieht, sondern auch im Beratungskontext einen internationalen Charakter hat.\\
	\\
\textbf{Allgemeine Arbeitsabläufe des IT-Consultings}\\ \\
	Nachdem die Arbeitskultur erklärt wurde, werden in diesem Kapitel allgemeine Abläufe des IT-Consultings detaillierter betrachtet. An dieser Stelle ist es unvermeidbar den Beratungsprozess exemplarisch zu zeigen, um die Feinheiten des Prozesses zu verstehen, die von der Arbeitskultur beeinflusst werden, um im Endeffekt Einschlüsse auf die Faktoren der Arbeitskultur zu bilden.\\ 
	IT-Consulting ist eine wichtige Art des Consultings in IT-Fragen eines Unternehmens. Das Wesen des IT-Consultings besteht im Allgemeinen darin, Unternehmen bei der Neustrukturierung der Anwendungslandschaften oder bei der Pflege der bestehenden Informationssysteme zu unterstützen. Während des gesamten Beratungsprozesses bleibt der Berater als externer Experte solange im Unternehmen bis die Probleme, die er mit seinem technischen Fachwissen zu lösen hat, nicht mehr existieren oder selbständig von den Mitarbeitern des Unternehmens gelöst werden können.\\
	Um den Beratungsprozess zu verdeutlichen wird nacholgend ein Beispielprozess aus der Praxis der IT-Beratung beschrieben. \\ Ein Online-Handelsunternehmen möchte ein BI-Standardsoftware einführen und die Daten für Analysezwecke aus dem bestehenden ERP-System laden, um die Kunden zu erkennen, die möglicherweise bald kündigen oder sich für Werbung von neuen Produkte eignen. Am Anfang jedes Prozesses muss dem Berater die Organisationsstruktur und die Geschäftsprozessabläufe des Unternehmens klar sein, um eine passende Lösung zu finden. Das IT-Beratungsunternehmen hat eine Standardsoftware im Einsatz, um die geschäftlichen Probleme eines Unternehmens mit Hilfe von Informationstechnologie zu lösen. Es gibt aber keine Standardlösung die für alle Unternehmensstrukturen passend ist, weil die Unternehmensstrukturen meistens heterogen sind. Nach dem erfolgreichen Vertragsabschluss zwischen Unternehmen und der Consultingfirma beginnt die Analysephase des Beratungsprozesses. Hier wird die Unternehmensstruktur des Online-Handelsunternehmen analysiert, bis man erkennt, wo die Software eingesetzt werden kann. Wichtige Aufgaben bestehen darin, Stellen wo Reibungen entstehen können zu finden, zu ermitteln welche Ressourcen zur Verfügung stehen und zu bestimmen, welches Informationssystem sich am besten für den Unternehmenszweck eignet. Es muss außerdem ein ständiges Feedback zwischen dem Berater und Projektleiter möglich sein.\\
	Nachfolgend wird der Ablauf des Beratungsprozesses intensiver beschrieben. Nach der Analysephase beginnt man der Konzepterstellung, indem für die Ideen und Pläne ein geeignetes Konzept erstellt wird. In der Folge beginnt die Umsetzungsphase, in welcher eine neue IT-Architektur aufgebaut oder die vorhandene ergänzt wird. Im unseren Beispiel wird die ERP-Lösung mit der BI-Lösung erweitert, die vorhandene Architektur bleibt erhalten. In dieser Phase können auch die andere Berater aufgerufen werden, falls es viele komplizierte Realisierungsmaßnahmen gibt.
	Nachdem das Informationssystem erfolgreich in die Unternehmensstruktur integriert ist, beginnen die Schulungsmaßnahmen, damit die Mitarbeiter des Unternehmens in der Lage sind mit diesem System umgehen zu können. Zum Schluss erfolgt die Wartungsphase und Intensität der Beratungsdienstleistung nimmt langsam ab. Diese Prozesskette kann in Form eines Lebenszyklus stattfinden.\\
	Diesen Ablauf kann man graphisch am folgenden Modell des ganzheitlichen Beratungsprozesses erkennen(siehe Abb. 5.1). Dieses Modell liefert uns die einzelnen Phasen der Beratungsdienstleistung eines Freelancers in dem Umfeld der Managementberatung \cite{MngmBerPhasen}.


\begin{figure}[htp]
\centering
\includegraphics[width=0.7\linewidth]{./images/beratungsproz}
\caption{Phasen des Beratungsprozesses eines Managementberaters, \cite{PhasenBeratungsprozess} }
\label{fig:beratungsproz}
\end{figure}
	
	\textbf{ Bedeutung der Arbeitskultur für IT-Consulting}\\ \\
	In wie weit ist es wichtig die Arbeitskultur für den Beratungsprozess zu betrachten? Anhand vom unseren Beispiel ist zu erkennen, dass die IT-Berater in jeder Phase der Softwareeinführung mit den Unternehmensvertretern kommunizieren sollen. Es ist wichtig, dass die Berater genug technisches Know-how mitbringen. Noch wichtiger sind die Soft Skills, die für erfolgreiche Geschäftsbeziehungen entscheidend sind. ``IT Business is People's Business``. Damit wird gemeint, dass der Erfolg von IT-Projekten sowohl von vertrauensvollen Verhandlungen als auch maßgeblich von der Kompetenz des Beraters abhängt. \cite{ITConsRu}
	Welche sozialen Fähigkeiten sind z.B. für einen us-amerikanischen IT-Berater besonders wichtig? Sind diese persönlichen Eigenschaften auch für die anderen Nationen von der Bedeutung? Unternehmensführung und IT-Berater müssen bei der Lösung des Problems einig werden. Der Berater muss ein Unternehmen für seine vorgeschlagene Lösung überzeugen. Muss man, um das Unternehmen zu überzeugen, nur eine gute Software anbieten und als vertrauenswürdiges Unternehmen am Markt agieren oder reichen diese Bedingungen beispielsweise in Indien nicht aus, weil der Berater möglicherweise aus anderer Kaste ist. Denn die Kastenzugehörigkeit hat in Indien bis heute kulturelle und soziale Auswirkungen auf viele Lebensbereiche \cite{KastensystemInd}.

	Für diese Arbeit ist wichtig zu wissen, wie die Arbeitskultur in ausgewählten Länder sich unterscheidet und in wie weit diese den Beratungsprozess beeinflussen kann.
	In den folgenden Kapiteln wird Arbeitskultur von ausgewählten Ländern (Russland Japan, USA, Deutschland) untersucht und zum Schluss werden einige interessante Fakten verglichen und diskutiert. 
\section{Teilaspekte}
	In diesem Kapitel werden die Teilaspekte der Arbeitskultur von ausgewählten Ländern vorgestellt. Einige Teilaspekte werden detaillierter beschrieben, um die Wichtigkeit dieser Aspekte für das IT-Consulting hervorzuheben. Für die bessere Übersicht und um später die die Teilaspekte in ausgewählten Ländern zu vergleichen, wird eine Matrix aufgestellt. Die Felder dieser Matrix bleiben zuerst leer und nach dem die einzelne Aspekte von den Ländern recherchiert und vorgestellt werden, wird die Matrix noch mal mit den ausgearbeiteten Feldern ausgefüllt, damit man daraus einen Vergleich ableiten kann. Die Recherche findet in 3 Sprachen (Deutsch, Englisch und Russisch) statt, um den Fokus der Recherche zu verbreiten.
	
\begin{table}[htp]
\begin{tabular}{|c|c|c|c|c|c|}
\hline  Aspekt/Land& Deutschland & USA & Russland & Japan & Indien \\ 
\hline 	Hierarchien  & ? & ? & ? & ? & ? \\ 
\hline  Gehalt& ? & ? & ? & ? & ? \\ 
\hline  Gesetze& ? & ? & ? & ? & ?  \\ 
%\hline  Grad des intuitiven Handelns& ? & ? & ? & ? & ? & ? \\ 
\hline  Kritikfähigkeit& ? & ? & ? & ? & ? \\ 
\hline  Team& ? & ? & ? & ? & ?\\ 
\hline  Entscheidungsfindung& ? & ? & ? & ? & ?  \\ 
\hline  Lebensstandard& ? & ? & ? & ? & ? \\ 
\hline  Pünktlichkeit& ? & ? & ? & ? & ?\\ 
\hline  Arbeitszeit, Urlaub, W-L-B& ? & ? & ? & ? & ?\\ 
\hline 
\end{tabular} 
\caption{Matrix der Arbeitskultur}
\end{table}
%Ende der Recherche-> Information für neue Matrix
%1)Hierarchien und Organisation->Hierarchien 2)Kundenverh weg 3)spezielle Rechtslage->Gesetze 4)Grad des intuitiven Handelns weg 5) Kritikfähigkeit nur bei Japan,De 6) Lebensumstände ->Lebensstandards ->Zeitmanagement in Form von Pünktlichkeit 7)Work-Life-Balance-> Arbzeit und Urlaub 8)Entsch.findung nur Russland
%
%Durch einer Recherchewerden einige Teilaspekte verändert. Dafür gibt es mehrere Gründe, bspw. waren die recherchierten Teilaspekte besser formuliert oder einige waren in ihrer Definition zu weit gefasst und müssten demzufolge zusammengefasst werden. Für bestimmte Teilaspekte gibt es keine Information, die durch Recherche in 3 verschiedenen Sprachen (Deutsch, Englisch und Russisch) nicht zu finden ist. An dieser Stelle besteht noch Forschungsbedarf. 
Dieses Kapitel hat im Vergleich zu den Kapiteln ``Markt`` und ``Bildung`` eine andere Struktur, indem nicht nach den Teilaspekten der Arbeitskultur gegliedert wird. Dafür wird hier eine Unterteilung des Kapitels nach Ländern vorgenommen und in diesen werden die Teilaspekte wie Hierarchien, Gesetze oder Team beschrieben.
Bevor die ausgewählten Ländern untersucht werden, definiert man in folgenden Kapiteln zuerst die wichtigsten Teilaspekte der Arbeitskultur.
\subsection*{Gehalt}
Gehalt ist einer der wichtigsten Faktoren für die angenehme Arbeitskultur. Diejenigen, die ein gutes und faires Gehalt bekommen sind motiviert, meist zufrieden mit ihrem Job und aus psychologischer Sicht sind sie sicher, dass sie für eigene Mühe ein gerechtes Gehalt bekommen. Demzufolge hat das Geld nicht nur die Tauschmittel-Funktion (es erlaubt uns, das zu kaufen, was wir zum Leben brauchen), sondern auch eine psychologische. Laut Täubner steht das Geld für Erfolg, Sicherheit, Anerkennung, Macht, Lebensqualität und Selbständigkeit \cite{GehaltBedeutungDE}.\\
Wenn die Arbeitnehmer im Gegensatz zum guten Verdienst mit ihrem Gehalt unzufrieden sind, dann gibt es meist kein Wohl am Arbeitsplatz. Es lässt sich doch sagen, dass das Geld nicht der wichtigste Faktor für gute Arbeitskultur ist. Laut der StepStone-Studie,in der rund 18.500 Fach- und Führungskräfte befragt wurden, um zu wissen was die Arbeitnehmer am meisten motiviert, steht das Geld nur auf dem 3.Platz nach dem guten kollegialen Arbeitsumfeld und dem Spaß am Arbeiten \cite{GehaltNR.3DE}.\\
 Schlussendlich lässt sich sagen, dass das Gehalt ein wichtiger Motivierungsfaktor ist, der allein zu schwach ist, um eine gute Arbeitskultur zu gewährleisten.
IT-Berater sind daher keine Ausnahme, wenn es um Gehalt geht. Berater arbeiten viel und möchten dafür auch entsprechend entlohnt werden. Die Arbeitszeiten von IT-Berater werden im nächsten Kapitel vorgestellt.\\
Interessant zu wissen ist noch, wie sich das Gehalt von IT-Beratern in den ausgewählten Ländern sich unterscheidet und in wieweit er deren Arbeitskultur beeinflusst.
\subsection*{Arbeitszeit, Urlaub, Work-Life-Balance}
Flexible Arbeitszeiten und viele Urlaubstage sind die wichtigsten Faktoren um die Work-Life-Balance zu bestimmen. Dazu zählen noch laut Feddersen: die Möglichkeit im Homeoffice zu arbeiten, Zeit für Fortbildung und Kinderbetreuung \cite{WLB}.\\
Work-Life-Balance ist heute kein leerer Begriff mehr. Das, was dahinter steht, gewinnt zunehmend an Bedeutung. Arbeitnehmern sowie Arbeitgebern wird immer bewusster, dass die Work-Life-Balance nicht nur die Balance zwischen Job und Privatleben, sondern auch ein wichtiger wirtschaftlicher Faktor ist. Die Work-Life-Balance beeinflusst die Arbeitskultur und auch die Arbeitsleistung sowie die Produktivität von Mitarbeitern, was für Arbeitgeber entscheidend ist, um das Gleichgewicht zwischen Berufs- und Privatleben für die Angestellten zu gewähren. Die Vorbeugung von Burnout-Erkrankungen ist dabei sicherlich ein weiterer, positiver Nebeneffekt. \cite{WLB} \\
Das Gleichgewicht zwischen Berufs- und Privatleben herzustellen, ist für IT-Berater nicht einfach. Denn IT-Berater sind ständig auf Reisen, arbeiten 60-70 Stunden pro Woche und diese Arbeit ist meist geistig  sehr anstrengend \cite{WLBbeiIT-Berater}. Für junge IT-Berater könnte es kein Problem sein, wie sieht es aus wenn man eine Familie und Kinder hat? Laut Robert Laube, der für Service Line ``Business Intelligence`` bei Avanade verantwortlich ist, erfordert die Work-Life-Balance sehr viel Selbstdisziplin \cite{WLBbeiIT-Berater}. Dazu gehört beispielsweise die Verbannung von E-Mails auf dem Handy oder ``[...] morgens mit den Kindern zu frühstücken und sie in die Schule und den Kindergarten zu bringen``, falls es die Zeit erlaubt \cite{WLBbeiIT-Berater}.
\subsection*{Lebensstandard}
Der Lebensstandard in den ausgewählten Ländern gehört auch zum wichtigen Teilaspekt der Arbeitskultur. Es ist vor allem wichtig das Gehalt von IT-Berater im Bezug zum Lebensstandard in einem bestimmten Land zu betrachten. Mit anderen Wörtern kann man den Lebensstandard als Ergänzung zum Gehalt oder als Ergänzung zu der Währung sehen, um das Gehalt in den ausgewählten Ländern objektiver zu vergleichen. An dieser Stelle ist es sinnvoll ein Beispiel zu erwähnen, um den Sachverhalt zu verdeutlichen. IT-Berater aus Deutschland verdienen durchschnittlich mehr als ihre russischen Kollegen, doch sind die Lebenshaltungskosten wie Nahrung, Miete, Kleidung in Russland geringer als in Deutschland.\\
Im Allgemeinen beschreibt der Lebensstandard den sozio-kulturellen Wohlstand von Personen im Verhältnis zu Vergleichspersonen innerhalb einer kulturellen Gemeinschaft. Bestimmt wird dazu die jeweilige Höhe der Lebensbedingungen bzw. die Befriedigung von materiellen und geistig-kulturellen Bedürfnissen. \cite{LbsWiki} \\
Natürlich unterscheiden sich die Faktoren, die den Lebensstandard in den ausgewählten Ländern beeinflussen. Hauptaugenmerk liegt jedoch darauf, wie der Lebensstandard und das Gehalt die Arbeitskultur von IT-Beratern beeinflusst, und nicht auf den internationalen Unterschied von einzelnen Lebensstandardindikatoren.
%\subsubsection{Team}
\section{Analyse der ausgewählten Teilaspekte}
	\subsection{Russland}	
	\textbf{Einleitung}\\ \\
	Der Aspekt-Markt spielt für die Arbeitskultur ebenfalls eine wichtige Rolle. Zwischen diesen Aspekte gibt es einige Zusammenhänge, wie das Gehalt oder die Arbeitszeiten von IT-Beratern.\\
	Russland ist ein Wachstumsmarkt mit Zukunft. Laut Holger Hirsch ist heute der damals geschützte russischer Markt offen für Exporte und Investitionen aus Deutschland. Dies gilt sowohl für IT-Beratungs-Unternehmen, die ihre Softwareprodukte in Russland integrieren auch für russische Manager, die bei der Informationstechnologie auf westliches Know-how setzen.\cite{ITConsRu}\\
	Da der Markt für IT-Beratung neu ist, muss man als IT-Berater aus Westen ganz viele Entscheidungen intuitiv treffen. Hier werden natürlich die Soft Skills des Beraters gefragt. Technische Fähigkeiten, funktionales Wissen und Branchen-Know-how sind selbstverständlich vorausgesetzt. ´´Der Zerfall der Sowjetunion und die Reformen im wirtschaftlichen und sozialen Gefüge Russlands haben einen erheblichen Einfluss auf die Arbeitskultur in gegenwärtigen russischen Organisationen´´\cite{ProzessbeglBerRU}.
	Deswegen überlappen sich die kulturellen- mit reformbedingten Faktoren der Arbeitskultur. Es ist daher sehr schwer den Ursprung dieser Faktoren zu unterscheiden. Am Beispiel des russischen Kollektivs könnte diese Überlappungen der russischen Kultur und sowjetischen Reformen ersichtlich werden. Es wird sich jedoch nachfolgend primär auf die Teilaspekte und der damit verbundenen Auswirkungen fokussiert, als auf die tiefer liegenden Ursachen.

	\textbf{Gehalt}\\ 
	\\
	Der russischer Senior-Consultant aus Moskau verdient im Mittel 3845 € monatlich. Das ist für russische Verhältnisse ein relativ hohes Gehalt. Zum Vergleich beträgt das durchschnittliche Gehalt in ganz Russland beim aktuellen Währungskurs 587,60 €  und speziell in Moskau 927.57 € \cite{RusGehAllgm}. Es gibt  in Russland sehr starke regionale Gehaltsunterschiede. Aus dem unten stehenden Diagramm kann man den Unterschied des monatlichen Gehalts für SAP-Berater ermitteln. Im Großen und Ganzen verdient man in beiden Metropolen Moskau und Sankt-Petersburg ca. das doppelte wie in anderen Großstädten wie Rostov, Wolgograd oder Omsk.
	\\
\begin{figure}[htp]
\centering
\includegraphics[width=0.7\linewidth]{./images/SAP-Berater_Gehalt_RU}
\caption{SAP-Berater Gehälter in Russland \cite{GehaltSAPBerRU}}
\label{fig:SAP-Berater_Gehalt_RU}
\end{figure}
\\
%Quelle:http://www.tadviser.ru/index.php/%D0%A1%D1%82%D0%B0%D1%82%D1%8C%D1%8F:%D0%A0%D1%8B%D0%BD%D0%BE%D0%BA_%D1%82%D1%80%D1%83%D0%B4%D0%B0_%D0%B2_%D0%A0%D0%BE%D1%81%D1%81%D0%B8%D0%B8_(%D0%98%D0%A2_%D0%B8_%D1%82%D0%B5%D0%BB%D0%B5%D0%BA%D0%BE%D0%BC)
	Ein Senior IT-Berater aus Deutschland verdient zum Vergleich durchschnittlich 6.250 € im Monat\cite{GehaltSAPBerDE}. Der Junior Berater im IT Umfeld ohne Projekterfahrung  verdient in Russland durchschnittlich 1200 € monatlich\cite{GehaltSAPBerRU}. Laut der russischen Arbeitsagentur ``rabota.ru`` steigt die Anfrage auf ERP-Systemen enorm und deswegen steigen auch die Gehälter für Spezialisten in diesem Umfeld. Die Anfänger im IT-Beratungsbereich sind beim Berufseinstieg dazu bereit fast kostenlos zu arbeiten, um wertvolle Erfahrungen im ERP-Bereich zu sammeln \cite{RusGehRabota}.
	Der deutsche Junior-Berater mit den gleichen Qualifikation und Erfahrung verdient ca. drei mal so viel (3750 €) \cite{GehaltSAPBerDE} als sein russischer Kollege.\\ \\
	\textbf{Team: %und Organisation: 
	Russisches Arbeitskollektiv gegen westlichen Team}\\
	\\
	``Das Arbeitskollektiv wurde in der sowjetischen 
	Epoche als das zentrale soziale Handlungsfeld propagiert. Es repräsentiert, dass die Geschlossenheit der Gruppe wichtiger als die Selbstverwirklichung der einzelnen Gruppenmitglieder ist. `` \cite{ProzessbeglBerRU}\\
	Gruppeninterne Konflikte blieben deshalb häufiger ungelöst oder wurden nicht diskutiert. Der Unterschied gegen dem westlichen Team besteht darin, dass russischer Kollektiv eine dauerhafte Einrichtung mit klar zugewiesenen Leitungskompetenzen ist, die vom Vorgesetzten häufig ausgeübt werden. Im Gegensatz zum russischen Kollektiv wird das westliche Team nur für die Dauer eines bestimmten Projektes eingerichtet und zeichnet sich durch die Gleichberechtigung aller Teammitglieder aus.\\
	So ein Kollektiv für Beratungszwecke ist demzufolge oft nicht flexibel und ist zu stark weisungsgebunden. Die Aufgaben im Kollektiv werden meistens vom Vorgesetzten vorgeschrieben. In unserem Fall von einem Projektleiter oder einem Manager. Solche Führungspersonen sind im IT-Beratungsfall oft an dem Büro gebunden und sind meistens in diesem Büro, während die Berater oft unterwegs bei den Kunden sind. Daher müssen die Entscheidungen intuitiv und unabhängig von dem Vorgesetzten getroffen werden. Die Tatsache, die Entscheidungen intuitiv zu treffen, spiegelt sich in dem Prinzip des russischen Kollektivs wieder.  ``Russische Organisationen zeichnen sich durch eine Konzentration von Macht auf die Führungskräfte aus. Ohne dem Vorgesetzten werden keine Entscheidungen getroffen``. \cite{ProzessbeglBerRU} \\
	Verlagerung von Entscheidungen auf die Mitarbeiter wird in Russland selten stattfinden, deswegen werden die Mitarbeiter von den Führungskompetenzen befreit und nehmen oft nur bloße Anweisungen entgegen. Für  den Beratungsprozess ist diese Tatsache ein großer Minuspunkt, weil die Berater das interdisziplinäres Wissen besitzen und  den vollen Handlungsspielraum in der IT-Beratungsbranche brauchen.
	Zu erwähnen wäre noch, dass die jungen Menschen von solcher Stereotypen weiter entfernt sind als die ältere ``sowjetische`` Generation. \\ \\
	\textbf{Gesetze: Personalauswahl und russische Gesetze}\\
	\\
	 Eine weitere wichtige Besonderheit ist die Personalauswahl. Häufig erfolgt die Auswahl von neuen Mitarbeitern nicht nach dem Kriterium der fachlichen Kompetenz. Sondern oft werden Arbeitsplätze unter Verwandten und 
	 Freunden vergeben. Es existieren fast keine etablierten Mechanismen von Angebot und Nachfrage auf dem Arbeitsmarkt. Vakanzen werden häufig nicht an den fachlich geeignetsten Bewerber vergeben, sondern an ``unseren Mann``(nash chelovek). Sinngemäße Übersetzung bedeutet, dass ``Unser Mann`` oder ```nash chelovek`` eine besondere, meist verwandtschaftliche Beziehung zum Unternehmensführer oder den Vorgesetzten des Unternehmens hat.\\
	 Ein weiteres für russische Arbeitskultur typisches Merkmal ist, dass die Gesetze, Bestimmungen und Regelungen keinen eindeutig verbindlichen Charakter haben. In Abhängigkeit von der Situation und den involvierten Personen, können Regeln oder Gesetze bewusst unberücksichtigt bleiben. Wie sich jedoch diese Abstufung darstellt ist nicht vorhersagbar. Das liegt auch daran, dass das russische Volk und die russischen Behörde sich einander nicht zutrauen. Die Strenge des Gesetzes wird oft in der Vernachlässigung der Gesetzgebung ausgeglichen. \cite{ProzessbeglBerRU}\\ \\ 
	 \textbf{Arbeitszeit, Urlaub, Work-Life-Balance}\\ %Work-life-Balance gehört dazu
	 \\
	 Die gesetzliche Wochenarbeitszeit in Russland beträgt 40 Stunden. Doch in den meisten Fällen wird diese Grenze weit überschritten. Die IT-Spezialisten arbeiten zwischen 10 und 11 Stunden am Tag in einem 5-Tage-Rhythmus \cite{ArbZeitRU}. 
	  Oft wird auch eine 6-Tage-Woche praktiziert. Zum Vergleich arbeiten deutsche IT-Berater oft weniger (siehe Kapitel-Deutschland ``Arbeitszeit und Urlaub``) und die wöchentliche Arbeitszeit der japanischen Kollegen (siehe Kapitel-Japan ``Arbeitszeit und Urlaub``) ist noch höher als in den anderen ausgewählten Ländern. 
	 In vielen Tarifverträgen in Deutschland beträgt der durchschnittliche Jahresurlaub 30 Arbeitstage. In Russland sind es dagegen nur 24 Tage. Der Arbeitstag beginnt bei russischen nicht produzierenden Firmen um 9 oder 10 Uhr \cite{ArbZeitRU}. Wenn ein IT-Berater um 10 Uhr mit seiner Arbeit beginnt, dann ist er vermutlich erst um 20-21 Uhr zu hause. Es ist dahe rzu vermuten, dass sich dies aufgrund der  hohen Arbeitszeiten auch auf private Bereiche wie Freundschaften, Familie und somit auf die Work-Life-Balance von russischen Beratern negativ auswirken könnte.   \\ 
	 Dies kann wiederum Auswirkungen auf die Arbeitsleistung und die Motivation haben.
	 \\ \\
	 \textbf{Pünktlichkeit und Reisen }\\
	 \\
	 Der IT-Berater-Beruf ist eine Tätigkeit, die mit einer höheren Reisebereitschaft verbunden ist. Beratung beinhaltet meist auch, beim Kunde vor Ort zu sein. In Deutschland sind die Berater ganz oft mit Autos unterwegs. Von einer deutschen Großstadt bis zur anderen braucht man durchschnittlich etwa 4-5 Stunden. In Russland gibt es 2 grundsätzliche Transportprobleme in Bezug auf das Consulting, die auf den Ersten Blick deutlich werden: Staus in Moskau und große Entfernungen zwischen den russischen Städten. Nachfolgend werden diese 2 Probleme näher erläutert. Das Land ist sehr groß und weit (es umfasst 11 Zeitzonen).
	 Zwischen Moskau und Nowosibirsk sind es ca. 4 Stunden nur Flugzeit plus 3 Stunden Zeitunterschied. Wenn ein Berater aus Moskau seinen Arbeitstag am Montag in Nowosibirsk beginnen möchte, muss er schon am Sonntag ausreißen. Die Reisen sind erschöpfend und werden von russischen Beratern, die z.B. eine Familie haben, nicht so gern angenommen. Für IT-Beratern mit Familie wird dadurch auch die Work-Life-Balance beeinträchtigt, da die Reisezeit gewöhnlich nicht als Arbeitszeit gewertet wird und somit von der Freizeit entbehrt werden muss.\\
	 Laut dem russischen Rating "Consulting research" aus 21 größten IT-Consulting-Unternehmen befinden sich 13 Unternehmen in Moskau\cite{RaitConsRU}.
	 Aus 100 größten russischen IT-Unternehmen befinden sich in Moskau 71 Firmen \cite{100BigITConsURU}. 
	 Moskau ist nicht nur die teuerste Hauptstadt der Welt und ein wirtschaftliches Zentrum des Landes, sondern auch ein strategischer Standort für IT-Unternehmen geworden.
	 Mit dem Wachstum der Stadt werden auch die Staus immer länger.``Nach Angaben des GPS-Navigationsanbieters TomTom ist Moskau Nummer eins unter den schlimmsten Stau-Städten der Welt\cite{MoskauStau1}.``
	 Da die Berater öfters unterwegs sind, ist es eine große Anstrengung in Moskau Auto zu fahren. Um von A nach B zu kommen wird ganz oft ein Metro benutzt. 
	 Deswegen ist es in Moskau ``erlaubt`` dem Berater sowie allen Geschäftsleuten ein Viertel bis halbe Stunde zum Meeting oder zum  Kunde zu spät zu kommen. Oft werden Staus als Ausrede, die auch akzeptiert wird, genutzt.\\
	 Allgemein zählt die Pünktlichkeit nicht zu den Stärken von Russen. Die Termine werden nicht immer eingehalten, E-Mails werden nicht sofort beantwortet und die Versprechungen sind nicht immer realistisch. Deswegen muss man als Berater diese Verzögerungen mit einplanen \cite{RusKnigge}.\\ \\
	 	 \textbf{Hierarchie und Entscheidungsfindung}\\
	 	 \\
	 Im Teilaspekt Organisation wurde erwähnt, dass in russischen Organisationen der Chef oder sogenannte Generaldirektor die alleinige Entscheidungskompetenz hat. So beschreibt auch der Sergey Frank, dass die Entscheidungskompetenzen in Russland nicht wie gewöhnt nach unten gehen, sondern nur der Geschäftsführer die Entscheidungsbefugnisse hat \cite{RuSFI}.
	 Deswegen kommunizieren die russischen IT-Berater oft nur mit dem Geschäftsführer des Unternehmens, was natürlich zur zeitlichen Verzögerungen im Projekt führen kann.\\
	 Gemäß ``Russland-Knigge`` \cite{RusKnigge} werden die Hierarchien in Russland oft  nicht eindeutig und nicht klar erkennbar. Die Berater müssen schon vor Beginn der Verhandlungen herausfinden, wer das entscheidende Wort hat. Damit wird keine Zeit durch unnötige Gespräche mit Personen, die möglicherweise keinen Einfluss auf den Verhandlungsverlauf haben, verloren.\\
	 Laut ``Businessknigge Russland`` sind ``die flachen Hierarchien nicht die Sache der Russen`` \cite{RusKnigge}. Es heißt, dass auf der Ebene in der Geschäftsstruktur unter dem Unternehmensführer  sehr viele anderen Personen sein könnte, die einerseits zum Geschäft gehören, anderseits ist die Zusammengehörigkeit dieser möglichen Personen zum Geschäft sowie deren Aufgabengebiet meistens nicht klar definiert ist. An dieser Stelle in der Unternehmensstruktur könnte oben beschriebener ``nash chelovek`` auftauchen.
	\\ \\
		 	 \textbf{Lebensstandard}\\ \\
	Laut den Ergebnissen von Umfragen, die von der russischen Agentur für Finanzforschungen (russ. Abk.: NAFI) in den Jahren 2004 bis 2011 durchgeführt wurden, hat sich die materielle Lage der russischen Bürger in den vergangenen  Jahren trotz der Krise von 2008 und 2009 allmählich verbessert.\\
	 Die sogenannte „Vormittelklasse“, zu der fast die Hälfte aller Einwohner Russlands gehört, bestätigte einen Wachstumstrend. Dabei handelt es sich um Bürger, die genug Geld für Lebensmittel und Kleidung haben, denen jedoch der Kauf von langlebigen Waren schwer fällt. Laut der Meinungsforschung betrug diese Menschengruppe vor sieben Jahren nur höchstens 30 Prozent der Bevölkerung. In den letzten 7 Jahren ist der Anteil der armen Bürger an der Landesbevölkerung von 14 \% in 2004 auf 5\% in 2011 gesunken. Auch die Gruppe der Einwohner, denen das Geld für die Nahrungsmittel, nicht aber für die Kleidung ausreicht, ist von 35 \% in 2004 auf 28 \% in 2011 zurückgegangen.\\
	 Trotzdem lässt sich sagen, dass es immer noch enorme finanzielle Unterschiede zwischen der reichen und armen Bevölkerung in Russland gibt. Das durchschnittliche Gehalt in Russland beim aktuellen Währungskurs beträgt 587,60 €, in Moskau hingegen 927.57 €. Es ist noch zu erwähnen, dass die kleinen Gehälter mit dem niedrigen Lebenshaltungskosten wie gesetzliche Ausgaben, Nahrung, Kleidung usw. ausgeglichen werden. 
	\subsection{Japan}
	\textbf{Pünktlichkeit, Kritik und Besonderheit der Kommunikation}\\
	\\
	Im Geschäftsleben sind die Japaner besonders pünktlich. Pünktlich bedeutet in diesem Fall 5 bis 10 Minuten vor einem Termin zu erscheinen. Selbst nur bei 5 Minuten Verspätung müssen sie an Ihren Geschäftspartner Bescheid geben, dass Sie sich verspäten. \cite{JPKnigge}. \\
	Die Kritik wird in Japan nicht direkt, sondern ausweichend und über den ``Umweg`` geäußert. Diese Kritikäußerung-Strategie wird auch von ausländischen Geschäftspartner erwartet. Eine Besonderheit in Japan hat die Bedeutung des Wortes ``Ja``. In einem Gespräch reagiert man mit ``Ja`` nicht auf die Zustimmung mit der Sache, sondern auf die Bestätigung des Zuhörers \cite{JPKnigge}. Ein lang-gesprochenes ```Ja`` bedeutet die Zustimmung vom bestimmten Sachverhalt.\\
	\\
	\textbf{Hierarchie und Rangordnung}\\
	\\
	Im japanischen Geschäftsleben spielt die Rangordnung eine wichtige Rolle.
	Beim Essen sitzt die wichtigste Person in der Mitte der Reihe. Je größer die Entfernung von ihr, desto geringer der Rang \cite{Business-KniggeFernost}.	
	Beim Meeting spricht der Ranghöchste zuerst. Bei der Begrüßung wird zuerst die Hand nicht der Frau gegeben, wie das in Deutschland üblich ist, sondern dem Ranghöchsten. \\
	\\
	\textbf{Arbeitszeit, Urlaub, Work-Life-Balance}\\
	\\
	Offiziell gilt in Japan die 40-Stunden-Woche. In der Realität sind Angestellten weit über dieses Stundenmaß hinaus bis spätabends 21 oder 23 Uhr im Büro. Wenn jemand eher nach Hause geht als sein Chef, muss er bei dem nächsten Mitarbeitergespräch mit Konsequenzen rechnen \cite{ArbZeitJP}. Den Arbeitsplatz eher als der Chef zu verlassen, zählt in Japan zu den schlechten Manieren im Geschäftsleben.
	Die Überschrift eines Online-Zeitungsartikels ``Im Japan arbeitet man sich bis zum Tode`` hat sich schon längst, und nicht unbedingt unberechtigter Weise, als japanisches Klischee etabliert. 
	16 Stunden am Tag im Büro, Schlafkammer, Mittagsschläfchen im Zug und unzählige Überstunden gehören zum modernen Arbeitsalltag in Japan. ``Schuld daran haben die Zeitarbeitsagenturen. Denn ca. ein Drittel der japanischen Arbeiterschaft besteht aus Zeitarbeitern``, sagt der amerikanische Journalist Jake Adelstein \cite{JPArbeit}. Im Gegensatz zu Deutschland, wo die Überstunden nicht immer bezahlt werden, werden in Japan die Überstunden zur zusätzlichen Geldquelle, ohne diese mancher Japaner nur schwer überleben könnte.
	Es wurde keine Quelle gefunden, um die Arbeitszeit der IT-Berater zu bestimmen. 
	In Deutschland hingegen  genießen die Arbeitnehmer 30 Tage bezahlten Urlaub. In den USA beträgt Jahresurlaub rund 2 Wochen. Offiziell sind in Japan im Durchschnitt 17 Freitage gewährt, allerdings werden nur 8 Tage im Krankheitsfall in Anspruch genommen. Weil während des Urlaubs keine Lohnfortzahlung stattfindet, muss ganz oft Urlaub unter einer Krankheit ``versteckt`` werden \cite{JPArbeitSozKultur}. Denn die Abwesenheit im Krankheitsfall wird vom Unternehmen akzeptiert und Lohnfortzahlung wird geschehen. Jedoch gibt es in Japan 16 gesetzliche Feiertage. Im Vergleich zu China mit 10 Feiertagen und Deutschland mit 20 Feiertagen, ist die Anzahl der Feiertage dafür durchschnittlich. \\
	\\
		\textbf{Gesetze: Steuer und Sozialausgaben}\\
		\\
		Steuern und Sozialabgaben sind in Japan niedriger als in Deutschland. Japanischer Mehrwertsteuer beträgt im Gegensatz zu Deutschland nur 10 \%.
		``Als Arbeitnehmer ist man in Firmen mit mehr als fünf Angestellten durch eine spezielle Krankenversicherung, einer sog. Employee Health Insurance abgesichert``. Jedoch muss man 10 bis 30 Prozent der Behandlungskosten aus eigener Tasche zahlen. Die Anteile für die Arbeitslosenversicherung übernimmt der Arbeitgeber \cite{ArbZeitJP}. \\
	\\
			\textbf{Lebensstandard}\\
			\\
		\\Der Lebensstandard in Japan ist vergleichbar mit dem mitteleuropäischen (sieh Abb.5.3).
		\begin{figure}[ht]
		\centering
		\includegraphics[width=0.7\linewidth]{./images/Lebensstandard-Pro-Kopf-Einkommen}
		\caption{Lebensstandard und Pro-Kopf-Einkommen \cite{LebensStd}}
		\label{fig:LebStdProKEink}
		\end{figure}\\
		Gemäß Miroslav Stimac sind die Ausgaben der Grundbedürfnisse in Deutschland höher als in Japan \cite[101]{Stimac2004}.
		Beispielsweise kostet Nahrung in Tokio durchschnittlich 2,36 mal mehr als in Deutschland. Viele japanische Städte leiden an Platznot, dies führt zu hohen Grundstückspreisen sowie Mietpreisen \cite[105]{Stimac2004}.
		Eine Familie mit 4 Mitgliedern im Großraum Tokio lebt oft in einer 60 qm-Wohnung \cite{ArbZeitJP}.\\	\\
		\textbf{Gehalt}\\ \\
		Das Einkommen in Japan hängt stark von dem Alter und der Betriebszugehörigkeit
		der Arbeitnehmer ab.  Laut den Angaben der Jobagentur ``CareerCross`` verdient ein japanischer IT-Berater durchschnittlich 8.000.532 Yen (56.804 €) \cite{GehaltJapan}.
		 \\ \\
			\textbf{Team}\\
			\\
		Die Bedeutung der Gruppe sowie ihr Erfolg orientiert sich, laut Michael Gehle,  nicht auf die individuellen Ziele der einzelnen Gruppenmitglieder, sondern auf einen gemeinsames Erfolg der Gruppe. Altruistisches Verhalten ist in japanischer Gesellschaft ausgeprägter als in Europa. Das Wohl des Einzelnen ist nach japanischer Auffassung vom Wohl seiner Kollegen abhängig. Mit anderen Worten bedeutet das, dass die erfolgreiche Zusammenarbeit der Gruppe eng mit der internen Zusammenarbeit der Gruppenmitglieder verbunden ist. Damit wird auch die Gruppe, ähnlich wie eine Familie betrachtet, welche Schutzfunktion übernimmt. In der nicht nur der Erfolg des gesamten Projektes, sondern auchdie Verantwortung an die Gruppenzugehörigen zu verteilen ist \cite[233]{3LaenderVergl}.\\ \\
		\textbf{Entscheidungsfindung} \\ \\
		Im Gegensatz zu Russland werden die Entscheidungsbefugnisse auch an die untere Hierarchieebene delegiert. Damit werden auch Managementanforderungen an die  Mitarbeitern niedriger Hierarchieebenen gestellt \cite[233]{3LaenderVergl}.\\
		Die Japaner haben  häufig sehr lange Entscheidungswege. Die Entscheidung wird nach top-down-Methode vom Chef angestoßen, dann verläuft die Akzeptanz der Entscheidung durch eine Organisationsspirale bis jeder Mitarbeiter diese Entscheidung wahrgenommen hat. In Deutschland wird die Entscheidungen hingegen eher formalistisch getroffen. Innerhalb des japanischen Teams gilt eine Gleichberechtigung und der Chef hat eine Rolle des Moderators. Falls während des Projektes Probleme aufgetreten sind, wird in Japan der Projektmanager häufig nicht kritisiert, sondern die Schuld wird an alle Projektmitglieder verteilt. In Deutschland dagegen trägt der Projektleiter oft die ganze Verantwortung für den Erfolg des Projektes.
%ab hier Korrektur
	\subsection{USA}
	\textbf{Team}\\
	\\
	In der USA, sowie in Deutschland, hängt der Erfolg sowie die Karriere des einzelnen Gruppenmitgliedes, nicht so stark vom Gruppenerfolg wie in Japan ab. Laut Michael Gehle orientieren sich Karriere und Qualifizierungsmaßnahmen mehr an einzelne Mitarbeiter. Deswegen richtet sich auch deren Verhalten eher nach individuellen und nicht nach kollektiven Zielen \cite[233]{3LaenderVergl}. 
	Teamkollegen werden in der USA, sowie in westlichen Ländern, nicht selten als Konkurrenten angesehen, weil die Entlohnung, sowie Kontrolle der Mitarbeiter an den individuellen Leistungen angehängt werden. 
	 \\
		\textbf{Arbeitszeit, Urlaub, Work-Life-Balance}\\
		\\
	Die Arbeitszeit in den USA beträgt in der Regal 40 Stunden pro Woche. 
	Ein Drittel aller US-Amerikaner arbeiten länger als 40 Stunden pro Woche.
	Laut der UN-Studie arbeiten US-Angestellte im Durchschnitt etwa 500 Stunden mehr als deutsche Arbeitnehmer \cite{ArbeitsumgUSA}. Die meisten US-Amerikaner haben nur 10 bezahlte Urlaubstage \cite{InfoUSArbVertr}.\\ \\
	\textbf{Gesetze: Besonderheiten in Arbeitsgesetzen}\\
		\\
		Die Lohnfortzahlung im Krankheitsfall beträgt in den USA nur 7 Tage pro Jahr. 
		Mitarbeiter im IT Bereich haben häufig keinen Anspruch auf Überstundenausgleich. Man kann allerdings häufig aber inoffiziell einen zusätzlichen Tag in der Woche frei nehmen, um die Überstunden auszugleichen. Als Überstundenausgleich dient auch am Ende des Jahres ein finanzieller Bonus \cite{InfoUSArbVertr}.
		Solche Vereinbarungen sind jedoch nicht gesetzlich geregelt und müssen zusätzlich werden.\\
		Die Kündigungsfrist beträgt, sowohl für Arbeitgeber als auch für 
		Arbeitnehmer, zwei Wochen. Aufgrund der kurzen Kündigungsfrist gibt es in den USA keine Probezeit.\\
		Im ersten Arbeitsjahr gibt es in den USA keinen Anspruch auf Urlaub. \cite{USA_Tipps}.
		\\ \\
	\textbf{Gehalt}\\
		\\
		Laut Firma ``indeed`` beträgt das Jahresgehalt des IT-Beraters (Technical Consultant`s in USA) rund 86.000 \$ (sieh Abb. 5.4).
		\begin{figure}[ht]
				\centering
				\includegraphics[width=0.7\linewidth]{./images/Techn_Cons_Sal}
				\caption{Jahresgehalt von IT-Consultants in USA}
				\label{fig:TechConsSal}
				\end{figure}\\
				%Wo ist Quelle???
		``Ein europäisches Gehalt in EUR ist ungefähr 1:1 vergleichbar mit einem 
		amerikanischen Gehalt in USD. Es wäre nicht richtig, den offiziellen Umrechnungskurs anzusetzen, da die Lebenshaltungskosten in USD in den USA mit den Lebenshaltungskosten in EUR in Europa vergleichbar sind`` \cite{InfoUSArbVertr}. In der IT-Branche wird oft ein Provision angeboten, die vom Erfolg des
		Projektes abhängt. Das Gehalt wird in zwei Raten ausbezahlt, in der Regel am 15. und am letzten Tag des Monats.\\ \\
	\textbf{Hierarchie} \\ \\
	Gemäß US-Internetportal ``hierarchystructure.com`` \cite{HierarchieUSA} gehört die US- Geschäftshierarchie zu den erfolgreichsten Geschäftshierarchien der Welt und wird von vielen Ländern als Vorbild gesehen, um wirtschaftliches Wachstum zu erzielen. Geschäftsstruktur und Hierarchie einer Organisation werden mit dem Zweck das Gruppenziel zu erreichen aufgebaut, wobei jeder einzelne Mitglied der Organisation unterstützt wird. Dabei werden die Positionen, Aufgaben und zugehörigen Mitarbeiterrollen klar definiert.
\begin{figure}[ht]
\centering
\includegraphics[width=0.7\linewidth]{./images/USA-Business-Hierarchy}
\caption{Geschäftshierarchie USA \cite{HierarchieUSA}}
\label{fig:USA-Business-Hierarchy}
\end{figure}

	\subsection{Deutschland}
	In allen beschriebenen Ländern unterscheidet sich der Formalisierungsgrad der Arbeitskultur erheblich. In Deutschland werden Verhaltensweisen und Standards der Arbeitskultur sehr formalisiert und in Form von gesetzlichen Vorschriften definiert. Dies führt zu einer starken Bürokratie. In Japan weist die Arbeitskultur auch einen hohen Formalisierungsgrad auf. Das hat jedoch weniger mit gesetzlichen Regelungen wie in Deutschland, sondern eher mit traditionellen Verhaltensweisen zu tun \cite[236]{3LaenderVergl}. Es gibt noch 2 weiteren Eigenschaften, welche die deutsche Arbeitskultur charakterisieren. Dazu zählen die sehr direkte Kommunikation und eine gründliche Planung.\\ \\
	\textbf{Pünktlichkeit}\\ \\
	Pünktlichkeit, sowie gründliche Planung, gehören zu den Stärken der Deutschen. IT-Consultants sind oft keine Ausnahme. Die Termine sollen gründlich geplant werden, bevor man zur Verhandlung kommt. Natürlich müssen die Termine zeitlich eingehaltenen werden. Es wird bei der Planung meist eine Pufferzeit eingerechnet, falls es trotz der aufwändigen Planung zu den zeitlichen Verzögerungen kommen soll.\\ \\
	\textbf{Arbeitszeit, Urlaub, Work-Life-Balance 
	} \\ \\
	Auch die  Trennung zwischen dem privaten und beruflichen Leben, zeichnet die  deutsche Arbeitskultur aus. Im Gegensatz zu China oder Russland, wo das Arbeitsteam oft als 2. Familie eingesehen ist, hat Deutschland an dieser Stelle einen formalistischen Charakter. Trotz allem was ein Team zusammen erlebt, heißen Teammitglieder unter einander ``Arbeitskollegen``. Laut ``Germany Trade and Invest``
	Gesellschaft für Außenwirtschaft und Standortmarketing sind Einladungen  von Geschäftspartner für private Aktivitäten in Deutschland eher selten, sofern die  Geschäftsbeziehungen nicht schon sehr lange bestehen \cite{ArbKulturDE}. Ob diese Tatsache eine besondere Auswirkung auf Beratungsgeschäft hat, ist nicht leicht differenzierbar. An dieser Stelle besteht ein Bedarf zur Forschung. \\
	Wie schon in vorigen Kapiteln erwähnt wurde, beträgt die gesetzlich geregelte Arbeitszeit in Deutschland 40 Stunden pro Woche. Montags bis Donnerstags sind die IT-Berater oft beim Kunde vor Ort und zeitlich sehr ausgelastet. Das hat damit zu tun, dass die Berater, wenn sie schon beim Kunde sind, versuchen möglichst viele Probleme zu lösen. Das führt nicht selten zu den Überstunden, die am Freitag entweder durch ein Meetings-Tag im Büro oder ein Selbststudium-Tag im Home-Office kompensiert werden. Am Freitag werden beispielsweise auftretende Probleme während des Beratung analysiert und neue Lösungen vorgeschlagen. Die Work-Life-Balance, was bei vielen Unternehmen heutzutage das Modethema ist, entspricht meistens nicht den Erwartungen der Junior-Berater \cite{JNRBer}. 
	Manche IT-Beratungsfirmen, meistens Großunternehmen, verteilen die Projekte nach einem Senioritätsprinzip. Das bedeutet, dass ältere Berater oder die Berater, die länger im Unternehmen sind und eine Familie besitzen, öfter in den lokalen Projekten tätig sind. Damit wird oft erreicht, dass solche Berater mehr Zeit für Freunde oder Familie haben, indem sie nicht so oft reisen, wie die anderen und sogar eine Möglichkeit haben in der Woche zu Hause übernachten.\\
	Der Arbeitstag beginnt in Deutschland im Vergleich zu den anderen Ländern wie in Russland, wo die meisten Büros erst ab 10 Uhr geöffnet sind, sehr früh. Bürozeiten ab 7 Uhr sind keine Seltenheit \cite{ArbKulturDE}. %Dieser Rhythmus wurde für meisten IT-Berater sehr gut eignen, weil man wegen der Einbindung an vielen Personen, alle    
	
	\textbf{Gehalt} \\ \\
	Die Gehälter der deutschen IT-Berater werden von vielen Faktoren beeinflusst. Dazu zählen die Größe, Branche und Region des Unternehmens, sowie die Noten des Studienabschlusses, das Tätigkeitsfeld, individuelle Fähigkeiten des IT-Consultants und andere Faktoren. Allgemein werden die IT-Berater sehr gut bezahlt. IT-Projektleiter, IT-Sicherheitsexperte, sowie IT-Berater verdienen am besten im deutschen IT-Umfeld \cite{VerdienstITinDE}.\\
	Senior IT-Berater aus Deutschland verdienen durchschnittlich 6.250 € monatlich. Der deutsche Junior-Berater ohne Erfahrung in der Beratungsbranche verdient ca. 3750 € im Monat .\cite{GehaltSAPBerDE} \\
	Die deutschen Berater, die ins Ausland geschickt werden, haben eine separate Vergütung über das gesetzliche Tagesgeld. \\ \\

	\textbf{Lebensstandard} \\ \\
	Laut der Rangliste für den Lebensstandard und Pro-Kopf-Einkommen von Jones und Klenow, die im Jahr 2010 für 134 Staaten veröffentlicht wurde, steht Deutschland auf dem 2. Platz nach der USA (sieh Kapitel Japan - Gesetze: Steuer und Lebensstandard \label{LebStdProKEink}). Trotz dieser Rangliste sinkt der Lebensstandard in Deutschland seit Euro-Einführung immer noch. Im Vergleich zu den südlichen europäischen Ländern wie Portugal oder Spanien, haben die Kernstaate der EU, zu denen z.B. Deutschland gehört, deutliche Verluste, nicht nur am Einkommen, hinnehmen müssen. \cite{SteigungDELebstd}.\\
	Um sich einen detaillierteren Eindruck zu verschaffen, ist es wichtig die Faktoren, welche den Lebensstandard  der deutsche Bevölkerung maßgeblich beeinflussen, zu bestimmen. Gemäß dem Online-Statistikportal statista, sind folgende Dinge (sieh Abb. 5.6) zu leisten, damit man einen akzeptablen Mindestlebensstandard in Deutschland hat.

\begin{figure} [hp]
\centering
\includegraphics[width=0.7\linewidth]{./images/FaktorenLebensstand}
\caption{Faktoren des Lebensstandards in Deutschland \cite{statistaDeLebnst}}
\label{fig:FaktorenLebensstand}
\end{figure}
%\textbf{Hierarchie}
\newpage
\section{Abschluss und Vergleich}
Zum Schluss werden wir die Arbeitskultur-Matrix überarbeiten, um festzustellen, welche Teilaspekte informationsreich sind und welche den Forschungsbedarf haben. Danach vergleichen wir einige Teilaspekte und signifikante Indikatoren der ausgewählten Länder, um Rückschlüsse auf die Arbeitskultur im IT-Beratungsgeschäft zu ziehen.
\begin{table}[htp]
\begin{tabular}{|c|c|c|c|c|c|}
\hline  Aspekt/Land& Deutschland & USA & Russland & Japan & Indien \\ 
\hline 	Hierarchien  & z & ja & ja & ja &  z \\ 
\hline  Gehalt& ja & ja & ja & ja & z \\ 
\hline  Gesetze& z & ja & ja & ja & z  \\ 
%\hline  Grad des intuitiven Handelns& ? & ? & ? & ? & ? & ? \\ 
\hline  Kritikfähigkeit& z & z & z & ja & z \\ 
\hline  Team& ja & ja & ja & ja & z\\ 
\hline  Entscheidungsfindung& x & x & ja & ja & z  \\ 
\hline  Lebensstandard& ja & ja & ja & ja & z \\ 
\hline  Pünktlichkeit& ja & z & ja & ja & z\\ 
\hline  Arbeitszeit, Urlaub, W-L-B& ja & ja & ja & ja & z\\ 
\hline 
\end{tabular} 
\caption{Matrix der Arbeitskultur 2}
\end{table}	\\
Um die überarbeitete Matrix richtig verstehen zu können, muss man zuerst die einzelnen Symbole definieren. Das Symbol ``Ja`` bedeutet, dass die Information über dem Teilaspekt bezüglich der IT-Beratung im internationalem Kontext vorhanden war und man daraus einen Vergleich über den gleichen Teilaspekt in ausgewählten Ländern machen kann. Das ``x`` bedeutet, dass man entweder die Information nicht vollständig hat oder, dass es keine eindeutige Beziehung zum unseren Thema gibt. Beispielsweise gibt es genug Information zu den deutschen Gesetzten, trotzdem gibt es hier keine Besonderheit, die den Beratungsprozess in maßgeblicher Weise beeinflusst. Das ``z`` bedeutet, dass aus zeitlichen Gründen keine Information gefunden wurde. Dazu gibt es mehrere Gründe wie bspw. die zeitliche Begrenzung des Projekts, sowie knappe Ressourcen in Form von Projektmitgliedern. Ein weiterer Grund besteht darin, dass es wichtiger war,  Prioritäten zu setzen und die Teilaspekte mit erhöhter Relevanz detaillierter zu betrachten.\\ \\
Einer wichtiger Teilaspekte ist das Gehalt der IT-Berater. Wenn wir die Gehälter von den Senior-Berater in den ausgewählten Ländern vergleichen, dann kann man vermutlich sagen, dass die Senior-IT-Berater aus Deutschland am meisten verdienen.
\begin{table}[htp]

\begin{tabular}{|c|c|c|c|c|}
\hline Land & Deutschland & USA & Japan &  Russland\\ 
\hline durch. Gehalt in € & 75.000 & 63.537 & 56.804 &  46.140\\ 
\hline 
\end{tabular} 
\caption{IT-Berater-Gehalt in ausgewählten Ländern}
\end{table}

Wenn wir das Gehalt in Beziehung mit dem Lebensstandard	in diesen Ländern betrachten. Dann können wir vermuten, dass die Lebenshaltungskosten in Russland  geringer sind als in Deutschland. Wenn wir das Gehalt und Lebensstandard ins Verhältnis setzen, dann ergibt sich ganz andere Gehälter-Ranking: Russland > Deutschland > USA > Japan.
\begin{figure}[ht]
		\centering
		\includegraphics[width=0.7\linewidth]{./images/Lebensstandard-Pro-Kopf-Einkommen}
		\caption{Lebensstandard und Pro-Kopf-Einkommen \cite{LebensStd}}
		\label{fig:LebStdProKEink}
		\end{figure}
Wenn unsere Vermutungen richtig sind, dann arbeiten IT-Berater aus Japan mit Abstand am meisten. Im Allgemeinen arbeiten IT-Berater in allen ausgewählten Ländern mehr als der Durchschnitt. Der Grund liegt an dem Beruf des Beraters und seinen spezifischen Eigenschaften wie Kundennähe, Reisebereitschaft usw. Um eine gute Work-Life-Balance zu erreichen, müssen Berater viel Selbstdisziplin haben, um die private Zeit ordentlich zu verwalten.\\
Wenn wir den Aspekt ``Gesetze`` betrachten, dann können wir abschließend sagen, dass es einige interessante Fakten in den ausgewählten Ländern gibt. Steuern und Sozialabgaben sind in Japan beispielsweise geringer als in Deutschland. In den USA gibt es Besonderheiten im Arbeitsgesetz. Es gibt keine Kündigungszeit aufgrund der kleinen Probezeit. In Russland gibt es eine Besonderheit bei der Personalauswahl(``unser Mann``) und die Gesetze haben meistens keinen eindeutig verbindlichen Charakter.\\
Wenn wir den Aspekt Team analysieren, dann kann man nachträglich sagen, dass einige Besonderheiten bei der Gruppenarbeit im Beratungsgeschäft in diesen Ländern gibt. In Russland, sowie in Japan wird das Team, in diesem Fall passt der Begriff ``Arbeitskollektiv`` am besten, von den historischen und kulturellen Eigenschaften des Volkes stark beeinflusst. Im Gegensatz zu diesen beiden Ländern, gibt es in den westlichen Ländern wie Deutschland und den USA eher ein modernes Team mit der Gleichberechtigung aller Teammitglieder.\\
Abschließend lässt sich sagen, dass es sehr wichtig war, den Aspekt der Arbeitskultur für den internationalen Beratungsprozess zu analysieren und zu vergleichen. Denn die Arbeitskultur beeinflusst nicht nur das Geschäftsleben von IT-Berater,, sondern auch deren Privatsphäre. Es hat sich auch viele Unterschiede sowie einige Gemeinsamkeiten zwischen ausgewählten Ländern bezüglich der Arbeitskultur im IT-Beratungsgeschäft herauskristallisiert. Es gab auch einige interessante Fakten, die für IT-Berater gut zu wissen sind, falls sie international agieren. Das wichtigste an dieser Stelle ist noch mal zu sagen, dass man die fremde Arbeitskultur als IT-Berater im internationalen Umfeld nicht unterschätzen soll.






%neu Matrix-die ausgefüllt wird, mit umformulierten Teilaspekten. Begrüdnung:
%Durch einer Recherchewerden einige Teilaspekte verändert. Dafür gibt es mehrere Gründe, bspw. waren die recherchierten Teilaspekte besser formuliert oder einige waren in ihrer Definition zu weit gefasst und müssten demzufolge zusammengefasst werden. Für bestimmte Teilaspekte gibt es keine Information, die durch Recherche in 3 verschiedenen Sprachen (Deutsch, Englisch und Russisch) nicht zu finden ist. An dieser Stelle besteht noch Forschungsbedarf. \\
%1)Hierarchien->Hierarchien(plus Organisation) 2)Kundenverh weg 3)spezielle Rechtslage->Gesetze 4)Grad des intuitiven Handelns weg 5) Kritikfähigkeit nur bei Japan,De 6) Lebensumstände ->Lebensstandards ->Zeitmanagement in Form von Pünktlichkeit 7)Work-Life-Balance-> Arbzeit und Urlaub(Work-Life-Balance wird ersichtlich) 8)tagesrythmus gehört zur Arbeitszeit und Urlaub


\chapter{Bildung, Ausbildung, Forschung}

\section{Einleitung}
Aufgrund des wissensintensiven Charakters des IT-Consultings, ist eine hoch entwickelte Bildungs- und Ausbildungsstruktur Grundvorraussetzung, für das erfolgreiche entstehen eines IT-Consulting Marktes. 
Gibt es beispielsweise nicht genügend Studienabgänger in einem Land, aber einen hohen Bedarf müssen ausländische Fachkräfte zugezogen werden. Die Bildungssituation spielt also auch für den Aspekt Markt eine wesentliche Rolle.

Um Unternehmen angemessen beraten zu können ist auf Seiten der Consultants ein hohes Bildungslevel nötig. Dies ist erforderlich um in angemessener Zeit ein tiefes und breitet IT-Fachwissen aufbauen zu können und die komplexen Zusammenhänge und Interdependenzen erkennen zu können. Außerdem muss auch ein hohes Level an betriebswirtschaftlicher Bildung bei den potenziellen Beratern vorhanden sein um die Unternehmen angepasst auf Ihre Wirtschaftliche Lage beraten zu können.
Die meisten deutschen Unternehmen fordern deswegen von ihren Bewerbern einen Studienabschluss (mind. Bachelor). Es existieren aber auch Länder wie zum Beispiel die Schweiz in der nur ein geringerer Teil der Schulabgänger ein Studium beginnt. In diesen Ländern ist es dann schwieriger eine Verbindung zwischen Studium und IT-Consulting herzustellen, da oftmals auch Personen ohne Studium akzeptiert werden.
Trotzdem stellt ein Studium in vielen Ländern aufgrund der wissensintensiven Tätigkeit des Beratens eine Grundvoraussetzung für die Beratung dar.
Deswegen wird im folgenden die Bildungs- und Ausbildungssituation in den einzelnen Ländern näher betrachtet.

Im Rahmen dieses Projektes beinhaltet jeder der Teilaspekte Aspekt Bildung, Ausbildung, Wissenschaft wiederum verschiedene Teilaspekte oder Kennzahlen die dann jeweils anhand von verschiedenen Ländern verglichen werden. Einige dieser Teilaspekte sind schwierig zu recherchieren, zu bestimmten Aspekten sind auch gar keine Aussagen möglich, dort besteht Forschungsbedarf. 

Teilweise ist es für die einzelnen Teilaspekte schwierig spezifische Daten zum IT-Consulting zu finden, da die Statistiken oft in Ingenieurwissenschaften und Wirtschaftswissenschaften unterteilen. IT-Consulting als interdisziplinär ausgerichtetes Fach ist deswegen oft schwer zuzuordnen. In solch einem Falle wird zum Vergleich eine Durchschnittsgröße aus beiden Studienfächern errechnet.

\section{Bildung}
Der Teilaspekt Bildung beschäftigt sich mit der allgemeinen Bildungssituation im einem Land. Dazu existieren verschiedene bereits Bewertungsverfahren unter anderem auch von der UNESCO. In der Veröffentlichung "World Data on Education" beschreibt die UNESCO die Bildungssituation in den einzelnen Ländern sehr detailliert.
Für die höher entwickelten Länder existieren auch vergleichende Studien der OECD. Die Schwierigkeit besteht jedoch im Vergleich mit Entwicklungsländern, da diese nicht mit zur OECD gehören. 

Einige Teilaspekte die den Bildungsstand eines Landes charakterisieren (diese stammen aus einem Brainstorming des Projektteames):
\begin{itemize} 
\item Anteil der Kinder die eine Schulausbildung machen können
\item Schulpflicht
\item Alphabetisierung
\item Bildungsausgaben
\item Bildungsausgaben Anteil am GDP
\item Anzahl der Schulabgänger ohne Abschluss
\item Qualität der Ausbildung (PISA Studien der OECD, Verschiedene Internationale Rankings)
\item Anteil der Studenten an einem Jahrgang
\item Anzahl der Absolventen Studium und Schule im Verhältnis zur Gesamtbevölkerung
\item Anteil der Hochschulen im Vergleich zur Fläche
\item Anteil der Hochschulen im Vergleich zur Bevölkerung
\item Unterstützung des Staates (vergleichbar mit BAFöG in Deutschland)
\item ... (weitere Punkte können zum Beispiel im Weltbildungsbericht der UNESCO gefunden werden)
\end{itemize}
Eine komplette Ausführung aller dieser Punkte für alle Länder würde den Rahmen dieser Arbeit sprengen, deswegen bezieht sich diese
dieser Teilaspekt in dieser Arbeit nur auf die Studien der UNESCO und der OECD und versucht daraus einen allgemeinen Bildungsstand eines Landes abzuleiten.

Im Folgenden werde einige ausgewählte Teilaspekt und Ihre Relevanz für den Bildungsstand eines Landes erläutert.

\subsection{Anzahl der Hochschulen im Verhältnis zur Fläche (Dichte)}
Die Anzahl der Hochschulen ist für den Bereich der universitären Forschung von großer Bedeutung. Diese prägen maßgeblich die Forschungslandschaft eines Landes mit. Auch für die Ausbildungssituation ist die Anzahl die Hochschulen relevant, denn um so mehr Hochschulen existieren, desto mehr ausgebildete Fachkräfte stehen der Wirtschaft (potenziell) zur Verfügung. 
Trotzdem ist es schwierig nur aufgrund der Anzahl der Hochschulen eine Aussage zu treffen, denn dabei wird die Größe eines Landes außer Acht gelassen. Intuitiv ist klar, das ein größeres Land (bei gleichem Entwicklungstand) im Vergleich zu einen Land das nur halb so groß ist mehr Hochschulen besitzen muss. Deswegen wird in dieser Arbeit die „Dichte“ der Hochschulen anhand des Verhältnissen von  Fläche des Landes zu der Anzahl der Hochschulen berechnet und als Kennzahl gebraucht.
Diese „Dichte“ besitzt jedoch auch einige Schwachstellen. So ist es mit dieser Kennzahl schwierig Länder mit großen dünn besiedelten Gebieten (z.B. Russland) mit kleinen hoch besiedelten Ländern (z.B. Deutschland) zu vergleichen. Allgemein wird einer sehr ungleichmäßigen Bevölkerungsverteilung im Vergleich zur Fläche keine Bedeutung beigemessen.

\subsection{Anzahl der Hochschulen im Verhältnis zur Bevölkerung}
Diese Kennzahl ist sehr ähnlich zum oben genannten Flächenverhältnis. Durch das Verhältnis von Bevölkerung zu Hochschulen ist jedoch die Besiedlungsdichte besser abgebildet als wenn nur die Fläche ins Verhältnis gesetzt wird.
Eine sehr ungleichmäßige Verteilung der Bevölkerung führt jedoch auch bei dieser Kennzahl zu Schwierigkeiten.

\subsection{Bildungsausgaben}
Die Ausgaben die Staat für die Bildung ansetzt sind ebenfalls ein wichtiger Teilaspekt, der die Bildung eines Landes charakterisiert.
Es existieren verschiedene Statistiken zu den Bildungsausgaben. So gibt es z.B. eine Erhebung, die Bildungsausgaben ins Verhältnis zum BIP eines Landes setzt. Diese Vergleichsgröße ist besser geeignet als nur den Geldbetrag für Bildung zu betrachten. Dies erkennt man indem man sich vor Augen führt, das bei gleichem Entwicklungsstand ein größeres Land (mit mehr Bevölkerung) höhere Bildungsausgaben haben muss. Trotzdem das größere Land absolut mehr für Bildung ausgibt, kann es jedoch relativ zur Bevölkerungsanzahl weniger ausgeben als das kleinere Land. Deswegen ist eine weitere Größe nötig zu der die Ausgaben ins Verhältnis gesetzt werden. Dies wird z.B. durch Statistiken zu den Pro-Kopf Ausgaben für Bildung oder durch das Verhältnis von BIP zu Bildungsausgaben erreicht.

\subsection{Unterstützung des Staates} 
Die Unterstützung des Staates für ein Studium spielt für einige Studenten eine wichtige Rolle, da sie es sich sonst nicht leisten könnten ein Studium zu beginnen. Durch staatliche Unterstützung wird es mehr Menschen ermöglicht zu studieren. Dadurch steigen auch die Absolventenzahlen an, somit stehen dem Arbeitsmarkt mehr Arbeitskräfte zur Verfügung.


 \section{Ausbildung}
Dieser Teilaspekt bezieht sich im Gegensatz zum eher allgemeinen Teilaspekt Bildung stärker auf das IT-Consulting. In diesem Abschnitt werden die spezifischen IT-Consulting Ausbildungsmöglichkeiten eines Landes näher beschrieben.
Da aber oftmals keine spezifischen Daten zu IT-Consulting Studiengänge oder Ausbildungen vorhanden sind (da es diese spezielle Ausrichtung oft nicht gibt) muss teilweise mit Fächergruppen gearbeitet werden.
Dadurch ergibt sich das Problem der Einordnung von IT-Consulting in die Fächergruppen Ingenieurwissenschaften und Wirtschaftswissenschaften/Rechtswissenschaften. Diese Zuordnung ist nicht möglich (interdisziplinäre Wissenschaft) deswegen wird mit einem Durchschnitt gearbeitet.

Teilaspekte die relevant sind:
\begin{itemize} 
\item Welche IT-Consulting Studiengänge existieren?
\item Welche anderen Studiengänge sind für IT-Consulting relevant?
\item Wie viele Absolventen gibt es in den relevanten Studiengängen?
\item Welche anderen Ausbildungsformen existieren für IT-Consulting?
\end{itemize}

Einige dieser Aspekte und ihre Relevanz für das IT-Consulting werden im folgenden erläutert.

\subsection{Relevante Studiengänge}
Wie bereits erwähnt erscheint es schwierig, IT Consulting zu einem Studienbereich oder einem Studiengang zuzuordnen. 
In Deutschland existieren vereinzelt aber auch Bachelor und Masterstudiengänge die Consulting im Namen tragen, so z.B. der Master Studiengang der Universität Hamburg. Ein genauerer Überblick über die in Deutschland, Österreich und der Schweiz verfügbaren spezifischen Bachelor Studiengänge wird in \cite{NissenKlaukDeelmannMohe201209} gegeben. Trotzdem stellen diese Absolventen nur einen kleinen Teil der benötigten Fachkräfte dar. 

Es erscheint logisch, das die Studienrichtung Wirtschaftsinformatik durch die interdisziplinäre Ausrichtung gut zum IT-Consulting passt. Auch die Studiengänge Betriebswirtschaftslehre sowie Informatik erscheinen für eine IT Consulting Karriere geeignet. Welche weiteren Studiengänge relevant sind lässt sich nur schwer ermitteln, das es keine Statistiken über die Herkunft der Berufseinsteiger im IT-Consulting gibt.

\subsection{Absolventen in relevanten Studiengängen}
Dieser Teilaspekt setzt natürlich eine Auswahl an relevanten Studiengängen oder Ausbildungen voraus. Die Anzahl der Absolventen wäre dann die potentielle Menge die dem Arbeitsmarkt zur Verfügung stehen würde.Werden besonders viele Ausländische Fachkräfte für die Beratung eingestellt, kann man auf einen Mangel an eigenen ausgebildeten Fachkräften schließen und somit auf einen höheren Bedarf.

\section{Forschung}
Der Bereich Forschung beschreibt die für das IT-Consulting relevanten Forschungen. Wiederrum ist es schwierig eine genaue Zuordnung von Forschungsaktivitäten in verschiedenen Fakultäten zur Beratungsbranche zu treffen. Zu diesem Teilaspekt besteht der größte Forschungsbedarf, da bisher nur wenig relevante Forschung betrieben wird.

Einige Teilaspekte zur Forschung sind beispielsweise:
\begin{itemize}
\item Verhältnis Forschung zur Praxis im IT-Consulting
\item Anzahl der wissenschaftlichen Veröffentlichungen die das IT Consulting betreffen
\item Forschungsgelder und Subventionen
\end{itemize}

Einige dieser Aspekte und ihre Relevanz für das IT-Consulting werden im folgenden erläutert.

\subsection{Anzahl der wissenschaftlichen Veröffentlichungen}
Dieser Teilaspekt betrifft die Anzahl der wissenschaftlichen Veröffentlichungen die sich mit IT-Consulting beschäftigen. 
Für Deutschland existiert eine vom statistischen Bundesamt veröffentliche Erhebung zur Anzahl und Fächerverteilung der Promovierenden.\cite{destatis}
Auch bei dieser Statistik ergibt sich die Schwierigkeit der Zuordnung eines Studienbereiches (Wirtschaftswissenschaften oder Ingenieurwissenschaftlich?) zum IT-Consulting. Ähnliche Statistiken über Bachelor und Master Arbeiten oder für andere Veröffentlichungen existieren für Deutschland nicht. 

\subsection{Verhältnis Praxis zu universitärer Forschung}
In einigen Veröffentlichungen wird IT-Consulting als rein praktische Tätigkeit betrachtet und somit der Sinn einer wissenschaftlichen Erforschung angezweifelt. Dem entgegen steht jedoch die Forschungsrichtung „Consulting Research“ (Volker Nissen). Consulting Research besitzt laut Nissen zwei Ziele: „Erstens, die wissenschaftliche Durchdringung des Themas Unternehmensberatung, wobei der von einzelnen Beratungsprojekten abstrahierende wissenschaftliche Erkenntnisgewinn im Mittelpunkt steht. Zweitens, die Übertragung wissenschaftlicher Theorien, Erkenntnisse und Methoden auf die unternehmerische Praxis mit dem Ziel, Aufgabenstellungen und Probleme im Umfeld von Beratungsprozessen und Beratungsunternehmen besser als bisher zu lösen. “
(Ausführliche Diskussion des Verhältnisses von Praxis zu Theorie auf S.23 ff)

Nissen erkennt auch ein „ Defizit an wissenschaftlicher Auseinandersetzung [...] mit Themen der IT-orientierten Beratung“. Dies erschwert es eine Aussage zum Forschungsstand im Consulting zu machen.

Diese Arbeit folgt der Auffassung von Nissen, dass eine wissenschaftliche Durchdringung des Consultings nutzbringend ist.

\subsection{Staatliche Forschungsgelder/Subventionen/Zuschüsse}
Dieser Teilaspekt beschäftigt sich mit der Höhe der Forschungsgelder die dem IT-Consulting zugeordnet werden können.Durch höhere Forschungsgelder können mehr Forscher eingestellt werden und mehr Zeit investiert werden, dies kann zu einer höheren Qualität der Ergebnisse führen. Forschungsgelder sind demnach relevant sowohl für die Menge der Forschungsarbeiten als auch für deren Qualität.
Für Deutschland existieren z.B. Statistiken die staatliche Forschungsgelder nach Fakultäten auflistet (sowohl für Universitäten als auch für Fachhochschulen). Erneut ist die Zuordnung des IT-Consultings zu einem der Fachbereiche Ingenieurwissenschaften oder Wirtschafts-/Rechtswissenschaften nur schwer möglich.


\section{Deutschland}
\subsection{Bildung}
\subsection{Ausbildung}
\subsection{Forschung}

\section{USA}
\subsection{Bildung}
\subsection{Ausbildung}
\subsection{Forschung}

\section{Indien}
\subsection{Bildung}
\subsection{Ausbildung}
\subsection{Forschung}







