\chapter{Schlussteil}
Im Schlussteil werden die erreichten Ergebnisse nochmal zusammengefasst. Dabei sollen Punkte die von den geplanten Ergebnissen noch abweichen aufgezeigt und Gründe dafür erläutert werden. Des weiteren soll ein Ausblick auf eine mögliche Fortsetzung der Arbeit gegeben werden. Dabei werden einige Vorschläge für weitere neue Arbeiten gegeben bzw. für Arbeiten, welche die Vermutungen die in dieser Arbeit gestellt werden verifizieren und weiter ausbauen. Zuerst sollen jedoch die erreichten Ergebnisse veranschaulicht werden.

\section{Zusammenfassung}
Im Großen und Ganzen wurde das Ziel einen guten Überblick über die Branche im internationalen Zusammenhang zu geben, durchaus erreicht. Jedoch ist dieser Überblick vor allem breit und damit nur sehr grob. Dadurch, dass das Gebiet Unternehmensberatung und IT-Consulting noch nicht sehr weit erforscht ist, müssen grundlegende Begriffe, Merkmale und Einflussfaktoren erst einmal herausgearbeitet werden. 
Außerdem wurde darauf geachtet, besonders im ersten Teil der Arbeit, welcher sich vor allem den grundlegenden Begriffen und Consultingarten widmet, mit möglichst vielen Quellen zu unterlegen. Es wurde dabei so vorgegangen, dass pro Begriffsdefinition mindestens drei qualitativ hochwertige Quellen zu einem Text synthetisiert wurden. Dies ermöglicht eine hohe und aktuelle Qualität der Begriffsdefinitionen, welche im allgemeinen Sprachgebrauch oft uneinig verwendet werden. 
Im zweiten Teil, dem Hauptteil der Arbeit, wurde sich um einen Ordnungsrahmen bemüht, welcher die Haupteinflussfaktoren der (IT-)Beratung charakterisiert. Diese unterliegen keiner trivialen Natur und sind nur durch eine sehr breite Recherche, ausgiebige Diskussion und gemeinsame Konsensfindung zu ermitteln. Die Hauptaufgabe bestand daher vor allem darin überhaupt erst einmal einen vollständigen Ordnungsrahmen für die Einordnung im internationalen Kontext zu finden, bevor einzelne Teilaspekte konkret analysiert werden können.
Dabei fiel auf, dass Teilaspekte äußerst vielfältig und schwer zugänglich sind bzw. durch bestimmte Faktoren ihre Aussagekraft verlieren können. Daher mussten Vorgehensweisen erdacht werden, die trotz der Komplexität und schlechten Zugänglichkeit der Informationen, vernünftige Rückschlüsse auf günstige bzw. ungünstige Faktoren liefern. Diese Vorgehensweisen müssen Kennzahlen, die eine Bewertung verwässern, mit einbeziehen. Da diese Methoden nur auf Vermutungen und vernünftige Überlegungen, welche wiederum auf Vereinfachung beruhen, basieren, bleiben die daraus gezogenen Schlüsse teilweise unsicher. Es resultieren daraus jedoch auch sehr signifikante Ergebnisse, die Aussagen mit sehr wahrscheinlichen Schlüsse zulassen. 
Aufgrund der Problematik dieser Unsicherheit, mussten einige Teilaspekte verworfen werden oder es wurde zusätzlicher Aufwand nötig um den Daten zu erheben und mögliche Schwächen zu diskutieren. 
Dieser erhöhte Aufwand, welcher aus der Komplexität der Thematik resultiert, hat dazu geführt, dass sich bei der geplanten konkrete Analyse nur auf wenige Teilaspekte bezogen werden konnte. Diese Teilaspekte wurden dafür möglichst vollständig erhoben und ausgewertet. Die arbeiteten konkreten Teilaspekte zeigen, wie eine solche erdachte Untersuchungsweise abläuft und durchaus Schlüsse zulässt, aber auch wieder neue Fragen aufwirft und teilweise die Bewertungen, erschwert.
Man kann durchaus sagen, dass das IT-Consulting, aufgrund seines interdisziplinären Charakters, besonders vielschichtig und eine Betrachtung im internationalen Kontext umso komplexer ist, jedoch eine Annäherung an das sehr unscharfe Arbeitsgebiet mit verschiedensten Methoden durchaus möglich ist. Um jedoch brauchbare Informationen zu beziehen, ist eine Fokussieren auf einzelne Teilaspekte notwendig.
\section{Ausblick}
Die Arbeit enthält einen Überblick über die relevanten Teilaspekte aus den Bereichen Markt, Bildung und Arbeitskultur, welche sich auf das (IT-)Consulting auswirken.
Es konnte jedoch aufgrund der entstandenen Vielfältigkeit von Teilaspekten, ein Großteil nicht erhoben werden. Es besteht daher zum einen noch weiterer Bedarf bei der Analyse von Teilaspekten und zum anderen bei der Verifizierung von Annahmen. Es werden auf Basis der Rechercheergebnisse Vermutungen angestellt, welche nun bei Bedarf konkret verifiziert und weiter analysiert werden können.
Eine weitere Aufgabe besteht darin, schließlich aus der Kombination der erhobenen Teilaspekte, weitere Schlüsse zu ziehen. Um diese Aufgabe zu bewältigen, werden jedoch erst noch mehr Informationen zu den Teilaspekten benötigt.
Um diese Aufgabe noch etwas einzuschränken, bietet es sich an, eine weitere Analyse nur auf Basis eines Vorhabens, wie z.B. der Ermittlung von Expansions- oder Outsourcingchancen, durchzuführen. So kann die Analyse gezielt nach einer Reihe der erarbeiteten Teilaspekten eingeschränkt und Schlussfolgerungen für dieses Vorhaben gezogen werden.
Des weiteren wird eine Ausdehnung auf weitere Länder, welche Chancenpotential haben oder gut dem Vergleich dienen, unter Umständen als sinnvoll erachtet.